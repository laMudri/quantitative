The ingredients of our fundamental lemma are perhaps well known
(relational interpretations, Kripke-indexing), but the value of our
framework lies in the generality of being able to choose $\mathcal{W}$
and its promonoidal structure, and the interpretion the $\oc_\rho$
modality as a relation transformer. Examples include:

\vspace{-0.6em}
 
\paragraph{Permutation Types} With the $\{0,1,\omega\}$ semiring, we
take the category $\mathcal{W}$ to consist of lists of some type of
keys, and permutations between them. The relation transformer is
defined as: $\oc_0 R~l = \top$, where $\top$ is the total relation,
$\oc_1 R~l = R~l$ and $R_\omega~R~l = (l = []) \land R~l$. With
suitable types of keys and lists of keys, the fundamental lemma states
that all programs are permutations. This result has already been
formalised in a one-off type system at
\url{https://github.com/bobatkey/sorting-types}.

\vspace{-0.6em}
 
\paragraph{Monotonicity Types} With $R$ the partially ordered semiring
with carrier $\{0,\uparrow,\downarrow,\equiv\}$ ordered
${\equiv} \leq {\uparrow},{\downarrow}$ and ${\uparrow}, {\downarrow} \leq 0$,
we take $\mathcal{W}$ to be the one-object, one-arrow category, and
define the relation transformer $\oc$ to be:
\begin{mathpar}
  \oc_0~R = \top

  \oc_\uparrow~R = R

  \oc_\downarrow~R = R^{op}

  \oc_\equiv~R = R \cap R^{op}
\end{mathpar}
If we let our base type be natural numbers with the relational
interpretation $R_{\mathrm{nat}}(n,n') \Leftrightarrow n \leq n'$,
then the fundamental lemma states that a program of type
$\ctxvar{x}{\mathrm{nat}}{\uparrow} \vdash t : \mathrm{nat}$ is
covariant (and similarly for contravariant and invariant).

\vspace{-0.6em}

\paragraph{Sensitivity Analysis} With the
$R = \mathbb{R} \cup \{\infty\}$ semiring, we let $\mathcal{W}$ be $R$
as well. The relation transformer is given by scaling. With a base
type of real numbers with relational intepretation
$R_{\mathrm{real}}~k~(x,x') \Leftrightarrow |x-x'| \leq k$, then the
fundamental lemma states that the usage annotations on the input
variables tracks the extent to which the program is sensitive to
changes in those variables.

\vspace{-0.6em}

\paragraph{Information Flow} With the $R = \mathcal{P}(L)$ semiring,
we again take $\mathcal{W} = R$, and let the relation transformer to
be
$\oc_l~R~l' = \{\top~\textrm{when }l \geq l'; R~\textrm{otherwise}\}$.
Then the fundamental lemma yields the same non-interference properties
as stated by Abadi et al. for the DCC \cite{abadi99core}.


% Local Variables:
% TeX-master: "quantitative"
% End:
