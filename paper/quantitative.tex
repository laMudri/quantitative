%% For double-blind review submission, w/o CCS and ACM Reference (max submission space)
%\documentclass[sigplan,review,anonymous]{acmart}\settopmatter{printfolios=true,printccs=false,printacmref=false}
%% For double-blind review submission, w/ CCS and ACM Reference
%\documentclass[sigplan,review,anonymous]{acmart}\settopmatter{printfolios=true}
%% For single-blind review submission, w/o CCS and ACM Reference (max submission space)
\documentclass[sigplan,review]{acmart}\settopmatter{printfolios=true,printccs=false,printacmref=false}
%% For single-blind review submission, w/ CCS and ACM Reference
%\documentclass[sigplan,review]{acmart}\settopmatter{printfolios=true}
%% For final camera-ready submission, w/ required CCS and ACM Reference
%\documentclass[sigplan]{acmart}\settopmatter{}


%% Conference information
%% Supplied to authors by publisher for camera-ready submission;
%% use defaults for review submission.
\acmConference[PL'18]{ACM SIGPLAN Conference on Programming Languages}{January 01--03, 2018}{New York, NY, USA}
\acmYear{2018}
\acmISBN{} % \acmISBN{978-x-xxxx-xxxx-x/YY/MM}
\acmDOI{} % \acmDOI{10.1145/nnnnnnn.nnnnnnn}
\startPage{1}

%% Copyright information
%% Supplied to authors (based on authors' rights management selection;
%% see authors.acm.org) by publisher for camera-ready submission;
%% use 'none' for review submission.
\setcopyright{none}
%\setcopyright{acmcopyright}
%\setcopyright{acmlicensed}
%\setcopyright{rightsretained}
%\copyrightyear{2018}           %% If different from \acmYear

%% Bibliography style
\bibliographystyle{ACM-Reference-Format}
%% Citation style
%\citestyle{acmauthoryear}  %% For author/year citations
%\citestyle{acmnumeric}     %% For numeric citations
%\setcitestyle{nosort}      %% With 'acmnumeric', to disable automatic
                            %% sorting of references within a single citation;
                            %% e.g., \cite{Smith99,Carpenter05,Baker12}
                            %% rendered as [14,5,2] rather than [2,5,14].
%\setcitesyle{nocompress}   %% With 'acmnumeric', to disable automatic
                            %% compression of sequential references within a
                            %% single citation;
                            %% e.g., \cite{Baker12,Baker14,Baker16}
                            %% rendered as [2,3,4] rather than [2-4].


%%%%%%%%%%%%%%%%%%%%%%%%%%%%%%%%%%%%%%%%%%%%%%%%%%%%%%%%%%%%%%%%%%%%%%
%% Note: Authors migrating a paper from traditional SIGPLAN
%% proceedings format to PACMPL format must update the
%% '\documentclass' and topmatter commands above; see
%% 'acmart-pacmpl-template.tex'.
%%%%%%%%%%%%%%%%%%%%%%%%%%%%%%%%%%%%%%%%%%%%%%%%%%%%%%%%%%%%%%%%%%%%%%


%% Some recommended packages.
\usepackage{booktabs}   %% For formal tables:
                        %% http://ctan.org/pkg/booktabs
\usepackage{subcaption} %% For complex figures with subfigures/subcaptions
                        %% http://ctan.org/pkg/subcaption

\usepackage{stmaryrd}
\usepackage{mathpartir}
\usepackage{amssymb}
\usepackage{cmll}
\usepackage{xcolor}


\newcommand{\ann}[2]{#1 : #2}
\newcommand{\emb}[1]{\underline{#1}}

\newcommand{\base}[0]{\iota}

\newcommand{\fun}[2]{#1 \multimap #2}
\newcommand{\lam}[2]{\lambda #1. #2}
\newcommand{\app}[2]{#1\ #2}

\newcommand{\excl}[2]{\oc_{#1} #2}
\newcommand{\bang}[1]{\operatorname{bang}\ #1}
\newcommand{\bm}[3]{\operatorname{bm}_{#1}(#2, #3)}

\newcommand{\tensorOne}[0]{1}
\newcommand{\unit}[0]{{*}_\otimes}
\newcommand{\del}[3]{\operatorname{del}_{#1}(#2, #3)}

\newcommand{\tensor}[2]{#1 \otimes #2}
\newcommand{\ten}[2]{(#1, #2)_{\otimes}}
\newcommand{\prm}[3]{\operatorname{pm}_{#1}(#2, #3)}

\newcommand{\withTOne}[0]{\top}
\newcommand{\eat}[0]{{*}_{\with}}

\newcommand{\withT}[2]{#1 \with #2}
\newcommand{\wth}[2]{(#1, #2)_{\with}}
\newcommand{\proj}[2]{\operatorname{proj}_{#1}(#2)}

\newcommand{\sumTZero}[0]{0}
\newcommand{\exf}[2]{\operatorname{ex-falso}_{#1}(#2)}

\newcommand{\sumT}[2]{#1 \oplus #2}
\newcommand{\inj}[2]{\operatorname{inj}_{#1}(#2)}
\newcommand{\cse}[4]{\operatorname{case}_{#1}(#2, #3, #4)}


\newcommand{\bind}[2]{\{#1\}#2}
\newcommand{\ctx}[2]{#1^{\color{red}#2}}
\newcommand{\ctxvar}[3]{#1 \stackrel{{\color{red}#3}}: #2}
\newcommand{\rescomment}[1]{{\color{red}#1}}

\newcommand{\sem}[1]{\llbracket #1 \rrbracket}

\begin{document}

%% Title information
\title%[Short Title]
       {Context Constrained Computation} %% [Short Title] is optional;
                                        %% when present, will be used in
                                        %% header instead of Full Title.
%\titlenote{Working title}             %% \titlenote is optional;
                                        %% can be repeated if necessary;
                                        %% contents suppressed with 'anonymous'
%\subtitle{Subtitle}                     %% \subtitle is optional
%\subtitlenote{with subtitle note}       %% \subtitlenote is optional;
                                        %% can be repeated if necessary;
                                        %% contents suppressed with 'anonymous'


%% Author information
%% Contents and number of authors suppressed with 'anonymous'.
%% Each author should be introduced by \author, followed by
%% \authornote (optional), \orcid (optional), \affiliation, and
%% \email.
%% An author may have multiple affiliations and/or emails; repeat the
%% appropriate command.
%% Many elements are not rendered, but should be provided for metadata
%% extraction tools.

%% Author with single affiliation.
\author{Robert Atkey}
%\authornote{with author1 note}          %% \authornote is optional;
                                        %% can be repeated if necessary
%\orcid{nnnn-nnnn-nnnn-nnnn}             %% \orcid is optional
\affiliation{
  % \position{Position1}
  \department{Computer and Information Sciences}              %% \department is recommended
  \institution{University of Strathclyde}            %% \institution is required
  % \streetaddress{Street1 Address1}
  % \city{City1}
  % \state{State1}
  % \postcode{Post-Code1}
%  \country{UK}                    %% \country is recommended
}
\email{robert.atkey@strath.ac.uk}          %% \email is recommended

%% Author with two affiliations and emails.
\author{James Wood}
% \authornote{with }          %% \authornote is optional;
                                        %% can be repeated if necessary
%\orcid{nnnn-nnnn-nnnn-nnnn}             %% \orcid is optional
\affiliation{
  % \position{Position2a}
  \department{Computer and Information Sciences}              %% \department is recommended
  \institution{University of Strathclyde}            %% \institution is required
  % \streetaddress{Street2a Address2a}
  % \city{City2a}
  % \state{State2a}
  % \postcode{Post-Code2a}
%  \country{UK}                   %% \country is recommended
}
\email{james.wood.100@strath.ac.uk}         %% \email is recommended


%% Abstract
%% Note: \begin{abstract}...\end{abstract} environment must come
%% before \maketitle command
% \begin{abstract}
%   Coeffects are a way of describing the context in which computation
%   takes place. Previous works on coeffect calculi have concentrated on
%   the additional capabilities bestowed upon programs operating in
%   known contexts. For example, a program that knows that its input are
%   streams has the ability to request the $n$th value in that
%   stream. In this work, we consider the converse: programs whose
%   contexts constrain the computations they an perform.

%   FIXME: more detail. And edit it down.
% \end{abstract}


%% 2012 ACM Computing Classification System (CSS) concepts
%% Generate at 'http://dl.acm.org/ccs/ccs.cfm'.
\begin{CCSXML}
<ccs2012>
<concept>
<concept_id>10011007.10011006.10011008</concept_id>
<concept_desc>Software and its engineering~General programming languages</concept_desc>
<concept_significance>500</concept_significance>
</concept>
<concept>
<concept_id>10003456.10003457.10003521.10003525</concept_id>
<concept_desc>Social and professional topics~History of programming languages</concept_desc>
<concept_significance>300</concept_significance>
</concept>
</ccs2012>
\end{CCSXML}

\ccsdesc[500]{Software and its engineering~General programming languages}
\ccsdesc[300]{Social and professional topics~History of programming languages}
%% End of generated code


%% Keywords
%% comma separated list
%\keywords{keyword1, keyword2, keyword3}  %% \keywords are mandatory in final camera-ready submission


%% \maketitle
%% Note: \maketitle command must come after title commands, author
%% commands, abstract environment, Computing Classification System
%% environment and commands, and keywords command.
\maketitle

\section{Introduction}
\label{sec:introduction}
In normal typed $\lambda$-calculi, variables may be used multiple
times, in multiple contexts, for multiple reasons, as long as the
types agree. The disciplines of linear types \cite{girard87linear} and
coeffects \cite{PetricekOM14,BrunelGMZ14,GhicaS14} refine this by
keeping track of how variables are used throughout a program. For
instance, we might track how many times a variable is used, or whether
it is used covariantly, contravariantly, or invariantly. Coeffects
give us a general framework of ``context constrained computing'',
where constraints on variables in the context tell us something
interesting about the computation being performed. Thus we put the
type information to work to tell us facts about programs that might
not otherwise be apparent.

We will present work in progress on capturing the ``intensional''
properties of programs via a family of Kripke indexed relational
semantics that refines a simple Set-theoretic semantics of
programs. The value of our approach lies in its generality. We can
accommodate the following examples:
\begin{enumerate}
\item Linear types that capture properties like ``all list
  manipulating programs are permutations''. This example uses the
  Kripke-indexing to track the collection of datums currently being
  manipulated by the program.
\item Monotonicity coeffects that track whether a program uses inputs
  co-, contra-, or in-variantly (or not at all).
\item Sensitivity typing, that tracks how sensitive the output of the
  program is in terms of changes to the input. This forms the core of
  systems for differential privacy \cite{reed10distance}.
\item Information flow typing, in the style of the Dependency Core
  Calculus \cite{abadi99core}.
\end{enumerate}
Through discusssion at the workshop, we hope to discover more
applications of our framework. In future work, we plan to extend our
framework with type dependency, and to explore the space of inductive
data types and elimination principles possible in the presence of
usage information.

The syntax and semantics we present here have been formalised in Agda:
\url{https://github.com/laMudri/quantitative/}. Formalisation of the
examples is in progress.

% Local Variables:
% TeX-master: "quantitative"
% End:


\section{Syntax}
\label{sec:syntax}
\subsection{Typing}
\begin{mathpar}
  \inferrule
  {(x : T) \in \Gamma}
  {\Gamma \vdash x \in T}

  \and

  \inferrule
  {\Gamma \vdash e \in S \\ S <: T}
  {\Gamma \vdash T \ni \emb{e}}
  \and
  \inferrule
  {\Gamma \vdash S \ni s}
  {\Gamma \vdash \ann{s}{S} \in S}

  \and

  \inferrule
  {\Gamma, x : S \vdash T \ni t}
  {\Gamma \vdash \fun{S}{T} \ni \lam{x}{t}}
  \and
  \inferrule
  {\Gamma \vdash f \in \fun{S}{T} \\ \Gamma \vdash S \ni s}
  {\Gamma \vdash \app{f}{s} \in T}

  \inferrule
  {\Gamma \vdash S \ni s}
  {\Gamma \vdash \excl{\rho}{S} \ni \bang{s}}
  \and
  \inferrule
  {\Gamma \vdash e \in \excl{\rho}{S} \\ \Gamma, x : S \vdash T \ni t}
  {\Gamma \vdash T \ni \bm{T}{e}{x}{t}}

  \and

  \inferrule
  { }
  {\Gamma \vdash \tensorOne \ni \unit}
  \and
  \inferrule
  {\Gamma \vdash e \in \tensorOne \\ \Gamma \vdash S \ni s}
  {\Gamma \vdash \del{S}{e}{s} \in S}

  \and

  \inferrule
  {\Gamma \vdash S \ni s \\ \Gamma \vdash T \ni t}
  {\Gamma \vdash \tensor{S}{T} \ni \ten{s}{t}}
  \and
  \inferrule
  {\Gamma \vdash e \in \tensor{S}{T} \\ \Gamma, x : S, y : T \vdash U \ni u}
  {\Gamma \vdash \prm{U}{e}{x}{y}{u} \in U}

  \and

  \inferrule
  { }
  {\Gamma \vdash \withTOne \ni \eat}
  \and
  \text{(no $\withTOne$ elim)}

  \and

  \inferrule
  {\Gamma \vdash S \ni s \\ \Gamma \vdash T \ni t}
  {\Gamma \vdash \withT{S}{T} \ni \wth{s}{t}}
  \and
  \inferrule
  {\Gamma \vdash e \in \withT{S_0}{S_1}}
  {\Gamma \vdash \proj{i}{e} \in S_i}

  \text{(no $\sumTZero$ intro)}
  \and
  \inferrule
  {\Gamma \vdash e \in \sumTZero}
  {\Gamma \vdash \exf{S}{e} \in S}

  \inferrule
  {\Gamma \vdash S_i \ni s}
  {\Gamma \vdash \sumT{S_0}{S_1} \ni \inj{i}{s}}
  \and
  \inferrule
  {\Gamma \vdash e \in \sumT{S}{T}
    \\ \Gamma, x : S \vdash U \ni s \\ \Gamma, y : T \vdash U \ni t}
  {\Gamma \vdash \cse{U}{e}{x}{s}{y}{t} \in U}
\end{mathpar}

\subsection{Resourcing}


\section{Semantics}
\label{sec:semantics}
\paragraph{Underlying Semantics} We give a standard semantics of types
and well typed terms into sets and functions. This semantics ignores
the usage information. For types, we have:
\begin{displaymath}
  \begin{array}{ll}
    \llbracket \base \rrbracket = A_\base \\
    \llbracket \fun{S}{T} \rrbracket = \llbracket S \rrbracket \rightarrow \llbracket T \rrbracket &
    \llbracket \excl{\rho}{S} \rrbracket = \llbracket S \rrbracket \\
    \llbracket \tensorOne \rrbracket = \llbracket \withTOne \rrbracket = \{*\} &
    \llbracket \tensor{S}{T} \rrbracket = \llbracket \withT{S}{T} \rrbracket = \llbracket S \rrbracket \times \llbracket T \rrbracket \\
    \llbracket \sumTZero \rrbracket = \{\} &
    \llbracket \sumT{S}{T} \rrbracket = \llbracket S \rrbracket \uplus \llbracket T \rrbracket \\
  \end{array}
\end{displaymath}
Contexts are interpreted as left-nested products. Terms are assigned
the usual semantics as functions $\sem{t} : \sem{\Gamma} \to \sem{S}$.

\paragraph{Usage-tracking semantics} To derive interesting properties
from our type system, we refine the set-theoretic semantics by
Kripke-indexed binary relations. This gives a fundamental lemma for
our system, that when instantitated in different ways captures the
examples in the introduction.

Our framework is parameterised by a category $\mathcal{W}$ of
\emph{possible worlds} that track how resources are distributed by
programs. To interpret resource separation, we assume that
$\mathcal{W}$ has \emph{symmetric promonoidal structure}: profunctors
$J : 1 \tobar \mathcal{W}$ and
$P : \mathcal{W} \times \mathcal{W} \tobar \mathcal{W}$ such
that $P \odot (J \times 1) \cong 1$, $P \odot (1 \times J) \cong 1$,
$P \odot (1 \times P) \cong P \odot (P \times 1)$, and
$P \cong P \odot (\pi_2 \times \pi_1)$, and the triangle, pentagon,
and hexagon laws hold\footnote{We don't need the laws to hold to prove
  the fundamental lemma.}.

We now assign to each type $T$ a functor
$\sem{T}^R : \mathcal{W}^{op} \to \mathrm{Rel}~\sem{T}$ that captures
a notion of $\mathcal{W}$-indexed ``indistinguishability''. To
interpret $\oc_\rho S$, we assume we are given a relation transformer
$\oc_A : R^{op} \to \mathrm{Rel}(A)^{\mathcal{W}^{op}} \to
\mathrm{Rel}(A)^{\mathcal{W}^{op}}$
that satisfies the axioms of a monoidal exponential comonad. The
interesting cases are for functions, $\otimes$-products and the
$\oc_\rho$ modality:
\begin{displaymath}
  \begin{array}{l}
    \llbracket \fun{S}{T} \rrbracket^R~w~(f,f') = \\
    \quad \forall x,y.~P(y,w)x \Rightarrow \forall s,s'.~\llbracket S \rrbracket^R y (s, s') \Rightarrow \llbracket
    T \rrbracket^R x (f~s, f'~s')
    \\
    \llbracket \tensor{S}{T} \rrbracket^R~w~((s, t), (s', t')) =\\
    \quad
    \exists x,y.~P(x,y)w \wedge \llbracket S \rrbracket^R x (s, s') \wedge
    \llbracket T \rrbracket^R y (t, t')
    \\
    \llbracket \excl{\rho}{S} \rrbracket^R~w~(s,s')=
    \oc_\rho \llbracket S \rrbracket^R~w~(s,s')\\
  \end{array}
\end{displaymath}
Contexts
$\ctxvar{x_1}{S_1}{\rho_1}, \ldots, \ctxvar{x_n}{S_n}{\rho_n}$ are
interpreted as if they were
$\sem{(\cdots(1 \otimes \oc_{\rho_1}S_1) \cdots \otimes \oc_{\rho_n}
  S_n)}$.
With these definitions, we can prove the following fundamental lemma
for our Kripke-indexed relational semantics.

% $R \times S$ and $R \uplus S$ are defined pointwise on relations.
% Particularly, there are two cases for $R \uplus S$:

% \begin{itemize}
%   \item $R(r, r')$ implies $(R \uplus S)(\mathrm{inl}~r, \mathrm{inl}~r')$.
%   \item $S(s, s')$ implies $(R \uplus S)(\mathrm{inr}~s, \mathrm{inr}~s')$.
% \end{itemize}

% We assume a family of natural transformations $\oc$ satisfying the following laws.

%   \begin{mathpar}
%     \rho \leq \pi \implies (\oc_\pi R~w \implies \oc_\rho R~w) \and
%     \oc_0 R~w \implies J w \and
%     \oc_{\rho+\pi} R~w~(a, b) \implies \exists x,y. P(x,y)w \wedge \oc_\rho R~x~a \wedge \oc_\pi R~y~b \and
%     \oc_1 R~w \iff R~w \and
%     \oc_{\rho \cdot \pi} R~w \iff \oc_\rho(\oc_\pi R)~w
%   \end{mathpar}

% The semantics of a context $\ctxvar{x_1}{S_1}{\rho_1}, \ldots, \ctxvar{x_n}{S_n}{\rho_n}$ is given by $\tensor{\llbracket \excl{\rho_1}{S_1}}{\tensor{\ldots}{\excl{\rho_n}{S_n}}} \rrbracket^R$.

% This indexed relational semantics gives us a family of logical relations.
% The fundamental lemma is as follows.

\begin{theorem}[Fundamental Lemma]
  \begin{displaymath}
    \ctx{\Gamma}{\Delta} \vdash t : T \implies \llbracket \ctx{\Gamma}{\Delta} \rrbracket^R w~(\gamma, \gamma') \implies \llbracket T \rrbracket^R w~(\sem{t}\gamma, \sem{t}\gamma')
  \end{displaymath}
\end{theorem}


% Local Variables:
% TeX-master: "quantitative"
% End:


\section{Example Instantiations}
\label{sec:examples}
The value of our framework lies in the multitude of examples of
context constrained computation that we can capture.

\paragraph{Monotonicity Types} Let $R$ the partially ordered semiring
with carrier $\{0,\uparrow,\downarrow,\equiv\}$ ordered
$\equiv \leq \uparrow,\downarrow$ and $\uparrow, \downarrow \leq 0$, with $\uparrow$ as the 
\begin{itemize}
\item The up-down 
\end{itemize}

\paragraph{Permutations} By using the right primitives, we can
instantiate our framework to create a language where every list manipulating program is a permutation. This 

In this case, we take $R$ to be the trivial
one-element semiring, 

\paragraph{Information Flow}

\paragraph{Metric Spaces}
\begin{itemize}
\item 
\end{itemize}


% Local Variables:
% TeX-master: "quantitative"
% End:



%% Acknowledgments
\begin{acks}                            %% acks environment is optional
                                        %% contents suppressed with 'anonymous'
  %% Commands \grantsponsor{<sponsorID>}{<name>}{<url>} and
  %% \grantnum[<url>]{<sponsorID>}{<number>} should be used to
  %% acknowledge financial support and will be used by metadata
  %% extraction tools.
  % This material is based upon work supported by the
  % \grantsponsor{GS100000001}{National Science
  %   Foundation}{http://dx.doi.org/10.13039/100000001} under Grant
  % No.~\grantnum{GS100000001}{nnnnnnn} and Grant
  % No.~\grantnum{GS100000001}{mmmmmmm}.  Any opinions, findings, and
  % conclusions or recommendations expressed in this material are those
  % of the author and do not necessarily reflect the views of the
  % National Science Foundation.
  James Wood is supported by a EPSRC award (FIXME).
\end{acks}


%% Bibliography
%\bibliography{bibfile}


%% Appendix
% \appendix
% \section{Appendix}

% Text of appendix \ldots

\end{document}
