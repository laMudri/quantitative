We start by giving a standard semantics of types and well typed (but not necessarily well resourced) terms into sets.

\begin{displaymath}
  \begin{array}{ll}
    \llbracket \base \rrbracket = A_\base \\
    \llbracket \fun{S}{T} \rrbracket = \llbracket S \rrbracket \rightarrow \llbracket T \rrbracket &
    \llbracket \excl{\rho}{S} \rrbracket = \llbracket S \rrbracket \\
    \llbracket \tensorOne \rrbracket = \llbracket \withTOne \rrbracket = \{*\} &
    \llbracket \tensor{S}{T} \rrbracket = \llbracket \withT{S}{T} \rrbracket = \llbracket S \rrbracket \times \llbracket T \rrbracket \\
    \llbracket \sumTZero \rrbracket = \{\} &
    \llbracket \sumT{S}{T} \rrbracket = \llbracket S \rrbracket \uplus \llbracket T \rrbracket \\
  \end{array}
\end{displaymath}

The semantics of contexts and terms are as usual.

Using this, we give a semantics of types $T$ into functors from $\mathcal{W}^{op}$ to $\mathrm{Rel}~\llbracket T \rrbracket$, where $\mathcal{W}$ is a symmetric promonoidal category and $\mathrm{Rel}~A$ is the category of binary relations on $A$.

\begin{definition}
  A \emph{symmetric promonoidal category} $(\mathcal{W}, J, P)$ is a category
  $\mathcal{W}$ and profunctors $J : 1 \nrightarrow \mathcal{W}$ and $P :
  \mathcal{W} \times \mathcal{W} \nrightarrow \mathcal{W}$ such that $P \odot (J
  \times 1) \cong 1$, $P \odot (1 \times J) \cong 1$, $P \odot (1 \times P)
  \cong P \odot (P \times 1)$, and $P \cong P \odot (\pi_2 \times \pi_1)$, such
  that the triangle, pentagon, and hexagon laws hold.
\end{definition}

\begin{displaymath}
  \begin{array}{llll}
    \llbracket \base \rrbracket^R & w & = & R_\base w \\
    \llbracket \fun{S}{T} \rrbracket^R & w & = & \lambda (f, f').~\forall x,y.~P(y,w)x \implies \phantom{X} \\
    \quad \forall s,s'.~\llbracket S \rrbracket^R y (s, s') \implies \llbracket
    T \rrbracket^R x (f~s, f'~s') \span \span \span \\
    \llbracket \tensorOne \rrbracket^R & w & = & \lambda (*, *).~J w \\
    \llbracket \tensor{S}{T} \rrbracket^R & w & = & \lambda ((s, t), (s', t')). \\
    \quad \exists x,y.~P(x,y)w \wedge \llbracket S \rrbracket^R x (s, s') \wedge
    \llbracket T \rrbracket^R y (t, t') \span \span \span \\
    \llbracket \withTOne \rrbracket^R & w & = & \top \\
    \llbracket \withT{S}{T} \rrbracket^R & w & = & \llbracket S \rrbracket^R w \times \llbracket T \rrbracket^R w \\
    \llbracket \sumTZero \rrbracket^R & w & \mathrm{impossible} \span \\
    \llbracket \sumT{S}{T} \rrbracket^R & w & = & \llbracket S \rrbracket^R w \uplus \llbracket T \rrbracket^R w \\
    \llbracket \excl{\rho}{S} \rrbracket^R & w & = & \oc_\rho \llbracket S
                                                   \rrbracket^R w \\
  \end{array}
\end{displaymath}

$R \times S$ and $R \uplus S$ are defined pointwise on relations.
Particularly, there are two cases for $R \uplus S$:

\begin{itemize}
  \item $R(r, r')$ implies $(R \uplus S)(\mathrm{inl}~r, \mathrm{inl}~r')$.
  \item $S(s, s')$ implies $(R \uplus S)(\mathrm{inr}~s, \mathrm{inr}~s')$.
\end{itemize}

We assume a family of natural transformations $\oc$ satisfying the following laws.

\begin{displaymath}
  \begin{array}{l}
    \rho \leq \pi \implies (\oc_\pi R~w \implies \oc_\rho R~w) \\
    \oc_0 R~w \implies J w \\
    \oc_{\rho+\pi} R~w~(a, b) \implies \exists x,y. P(x,y)w \wedge \oc_\rho R~x~a \wedge \oc_\pi R~y~b \\
    \oc_1 R~w \iff R~w \\
    \oc_{\rho \cdot \pi} R~w \iff \oc_\rho(\oc_\pi R)~w
  \end{array}
\end{displaymath}

The semantics of a context $\ctxvar{x_1}{S_1}{\rho_1}, \ldots, \ctxvar{x_n}{S_n}{\rho_n}$ is given by $\tensor{\llbracket \excl{\rho_1}{S_1}}{\tensor{\ldots}{\excl{\rho_n}{S_n}}} \rrbracket^R$.

This indexed relational semantics gives us a family of logical relations.
The fundamental lemma is as follows.

\begin{displaymath}
  \ctx{\Gamma}{\Delta} \vdash t : S \implies \llbracket \ctx{\Gamma}{\Delta} \rrbracket^R w~(\gamma, \gamma') \implies \llbracket S \rrbracket^R w~(s, s')
\end{displaymath}


% Local Variables:
% TeX-master: "quantitative"
% End:
