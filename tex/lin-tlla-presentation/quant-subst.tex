\documentclass[fleqn]{beamer}

\usepackage{booktabs}
\usepackage{subcaption}

\usepackage{stmaryrd}
\usepackage{mathpartir}
\usepackage{amssymb}
\usepackage{cmll}
\usepackage{xcolor}
\usepackage{paralist}
\usepackage{amsmath}
\usepackage{amsthm}
\usepackage{mathrsfs}
\usepackage{mathtools}
\usepackage{tabularx}
\usepackage{tikz-cd}
\usepackage[conor]{agda}
\usepackage[style=alphabetic,citestyle=authoryear]{biblatex}
\addbibresource{../quantitative.bib}

\usetikzlibrary{positioning}
\usetikzlibrary{arrows}

\DeclareFontFamily{U}{cal}{}
\DeclareFontShape{U}{cal}{m}{n}{<->cmsy10}{}
\DeclareSymbolFont{rcal}{U}{cal}{m}{n}
\DeclareSymbolFontAlphabet{\mathcal}{rcal}

\def\newelims{1}
\def\debruijn{1}
\def\multnotation{1}
\ifx\newelims\undefined
  \def\newelims{0}
\fi

\newcommand{\bind}[2]{\{#1\}#2}
%\newcommand{\ctx}[2]{{#1}^{\rescomment{#2}}}
\newcommand{\ctx}[2]{{#2}^{#1}}
\newcommand{\ctxvar}[3]{#1 \stackrel{\rescomment{#3}}: #2}
%\newcommand{\rescomment}[1]{{\color{red}#1}}
\newcommand{\rescomment}[1]{{#1}}


\newcommand{\ann}[2]{#1 : #2}
\newcommand{\emb}[1]{\underline{#1}}

\newcommand{\base}[0]{\iota}

\newcommand{\fun}[2]{#1 \multimap #2}
\newcommand{\lam}[2]{\lambda #1.~#2}
\newcommand{\app}[2]{#1~#2}

\newcommand{\excl}[2]{\oc_{#1} #2}
\newcommand{\bang}[1]{\operatorname{bang} #1}
\newcommand{\bm}[4]{\if\newelims0%
\operatorname{bm}_{#1}(#2, \bind{#3}{#4})%
\else%
\mathrm{let}~\bang{#3} = #2~\mathrm{in}~#4 : #1%
\fi}

\newcommand{\tensorOne}[0]{1}
\newcommand{\unit}[0]{{*}_\otimes}
\newcommand{\del}[3]{\if\newelims0%
\operatorname{del}_{#1}(#2, #3)%
\else%
\mathrm{let}~\unit = #2~\mathrm{in}~#3 : #1%
\fi}

\newcommand{\tensor}[2]{#1 \otimes #2}
\newcommand{\ten}[2]{(#1, #2)_{\otimes}}
\newcommand{\prm}[5]{\if\newelims0%
\operatorname{pm}_{#1}(#2, \bind{#3, #4}{#5})%
\else%
\mathrm{let}~\ten{#3}{#4} = #2~\mathrm{in}~#5 : #1%
\fi}

\newcommand{\withTOne}[0]{\top}
\newcommand{\eat}[0]{{*}_{\with}}

\newcommand{\withT}[2]{#1 \with #2}
\newcommand{\wth}[2]{(#1, #2)_{\with}}
\newcommand{\proj}[2]{\operatorname{proj}_{#1} #2}

\newcommand{\sumTZero}[0]{0}
\newcommand{\exf}[2]{\operatorname{ex-falso}_{#1}(#2)}

\newcommand{\sumT}[2]{#1 \oplus #2}
\newcommand{\inj}[2]{\operatorname{inj}_{#1} #2}
\newcommand{\cse}[6]{\if\newelims0%
\operatorname{case}_{#1}(#2, \bind{#3}{#4}, \bind{#5}{#6})%
\else%
\mathrm{case}~#2~\mathrm{of}~\inj{0}{#3} \mapsto #4; \inj{1}{#5} \mapsto #6%
\fi}


\newcommand{\typed}[1]{\mathit{#1t}}
\newcommand{\resourced}[1]{\mathit{#1r}}


\newcommand{\sem}[1]{\llbracket #1 \rrbracket}

%\def\tobar{\mathrel{\mkern3mu  \vcenter{\hbox{$\scriptscriptstyle+$}}%
%                    \mkern-12mu{\to}}}
\newcommand\tobar{\mathrel{\ooalign{\hfil$\mapstochar\mkern5mu$\hfil\cr$\to$\cr}}}

\newcommand{\foo}[2]{\rescomment{\mathcal #1_#2}}

\title{A Linear Algebra Approach to Linear Metatheory}
\author{James Wood\inst{1} \and Bob Atkey\inst{1}}
\institute{\inst{1}University of Strathclyde}
\date{TLLA/Linearity, June 2020}

\begin{document}
\frame{\titlepage}
\begin{frame}
  \frametitle{Outline}
  \begin{itemize}
    \item Formalised syntactic metatheory for linear $\lambda$-calculus
    \item \ldots in an intuitionistic style
    \item \ldots using standard linear algebra constructs
  \end{itemize}
\end{frame}
\begin{frame}
  \frametitle{Algebra}
  \begin{itemize}
    \item Fix a partially ordered semiring\footnote{We can get away with a skew
      semiring --- a mild generalisation.}
      $(\rescomment{\mathscr R}, \subres, 0, +, 1, *)$.
      \begin{itemize}
        \item $(\rescomment{\mathscr R}, \subres)$ is a partial order.
          ``constraint strengthening''
        \item $(\rescomment{\mathscr R}, 0, +)$ is a commutative monoid.
          ``accumulation''
        \item $(\rescomment{\mathscr R}, 1, *)$ is a monoid. ``modal
          modification''
        \item $0$ annihilates, $+$ distributes, $+$ and $*$ are monotonic.
      \end{itemize}
    \item Natural deduction
      $M : \ctxvar{x_1}{A_1}{r_1}, \ldots, \ctxvar{x_n}{A_n}{r_n} \vdash B$
    \item Let
      $\resctx R = \begin{pmatrix}
        \rescomment{r_1} & \cdots & \rescomment{r_n}
      \end{pmatrix}$;
      $\Gamma = \begin{pmatrix}A_1 & \cdots & A_n\end{pmatrix}$. \\
      Then $M : \ctx{\Gamma}{R} \vdash B$.
    \item $M$ is an intrinsically typed derivation of
      $\ctx{\Gamma}{R} \vdash B$.
  \end{itemize}
  \begin{columns}
    \begin{column}{0.5\textwidth}
      \centering
      \begin{tikzpicture}[scale=0.5,baseline={(0,1)}]
        \node (omega) at (0,0) {$\rescomment\omega$};
        \node (zero) at (-2,2) {$\rescomment 0$};
        \node (one) at (2,2) {$\rescomment 1$};
        \draw (zero) -- (omega)
              (one) -- (omega);
      \end{tikzpicture} \\
      \begin{tabular}{rcl}
        $\rescomment 1$ & $\sim$ & linear variable \\
        $\rescomment\omega$ & $\sim$ & intuitionistic variable
      \end{tabular}
    \end{column}
    \begin{column}{0.25\textwidth}
      \centering
      \begin{tikzpicture}[scale=1,baseline={(0,0)}]
        \node (any) at (0,1) {$\rescomment\wn$};
        \node (up) at (-1,0) {$\rescomment\uparrow$};
        \node (down) at (1,0) {$\rescomment\downarrow$};
        \node (const) at (0,-1) {$\rescomment\equiv$};
        \draw (any) -- (up)
              (any) -- (down)
              (up) -- (const)
              (down) -- (const);
      \end{tikzpicture}
    \end{column}
    \begin{column}{0.25\textwidth}
      \begin{flalign*}
        0 &:= \rescomment\wn &\\
        1 &:= \rescomment\uparrow &\\
        + &:= \wedge &\\
        \rescomment\downarrow * \rescomment\downarrow &:=
        \rescomment\uparrow &
      \end{flalign*}
    \end{column}
  \end{columns}
\end{frame}
\begin{frame}
  \frametitle{Type system}
  \begin{block}{Variables}
    \begin{itemize}
      \item $\Gamma \ni A$ --- variables in $\Gamma$ of type $A$
      \item
        \(
        \inferrule
        {i : \Gamma \ni A \\ \resctx R \subres \basis i}
        {\ctx{\Gamma}{R} \lvar A}
        \)
    \end{itemize}
  \end{block}
  \vspace{-1em}
  \begin{block}{Terms}
    \vspace{-2em}
    \begin{mathpar}
      \inferrule*[right=var]
      {x : \ctx{\Gamma}{R} \lvar A}
      {x : \ctx{\Gamma}{R} \vdash A}
      \and
      \inferrule*[right=lam]
      {\bind{x}M : \ctx{\Gamma}{R}, \ctxvar{x}{A}{r} \vdash B}
      {\lam{x}{\bind{x}M} : \ctx{\Gamma}{R} \vdash \fun{\rescomment rA}{B}}
      \and
      \inferrule*[right=app]
      {M : \ctx{\Gamma}{P} \vdash \fun{\rescomment rA}{B}
        \\ N : \ctx{\Gamma}{Q} \vdash A
        \\ \resctx R \subres \resctx P + \rescomment r\resctx Q
      }
      {\app{M}{N} : \ctx{\Gamma}{R} \vdash B}
    \end{mathpar}
    \vspace{-2em}
  \end{block}
  \begin{itemize}
    \item Nameless --- de Bruijn indices
    \item We can also have
      $\excl{r}{A}$,
      $\tensorOne{}$, $\tensor{\rescomment pA}{\rescomment qB}$,
      $\withTOne{}$, $\withT{A}{B}$,
      $\sumTZero{}$, $\sumT{\rescomment pA}{\rescomment qB}$,
      bidirectional typing constructs\ldots
  \end{itemize}
\end{frame}
\begin{frame}
  \frametitle{Syntax-to-syntax traversals}
  \begin{itemize}
    \item Replacing variables by:
      \begin{itemize}
        \item Renaming --- \emph{variables} in the new context
        \item Substitution --- \emph{terms} in the new context
      \end{itemize}
    \item Substitution generalises renaming, but its proof depends on renaming.
    \item Our traversal lemma\footfullcite{rensub05} ($(-)\rescomment\Psi$ a
      linear map):
      \begin{mathpar}
      \inferrule*[lab=trav]
      {\forall C.~(j : \Delta \ni C) \to
      (\basis j\rescomment\Psi)\Gamma \kitrel C
      \\ \resctx P \subres \resctx Q\rescomment\Psi
      \\ \ctx{\Delta}{Q} \vdash A}
      {\ctx{\Gamma}{P} \vdash A}
      \end{mathpar}
  \end{itemize}
\end{frame}
\begin{frame}
  \frametitle{Kits\only<2-3>{ --- }\only<2>{variables}\only<3>{terms}}
  \newcommand\myrel{\only<1>\kitrel\only<2>\lvar\only<3>\vdash}
  Thingies replacing variables should be $\ctx{\Gamma}{R} \myrel A$ such that:
  \begin{itemize}
    \item We respect weakening of constraints.
      \[
      \resctx P \subres \resctx Q \implies
      \inferrule
      {\ctx{\Gamma}{Q} \myrel A}
      {\ctx{\Gamma}{P} \myrel A}
      \:\TirName{psh}
      \]
    \item We can turn usage-tracked variables into thingies.
    \item We can turn thingies into terms.
    \item We can introduce new variables \emph{at $0$-use}.
  \end{itemize}
  \begin{mathpar}
    \inferrule*[right=vr]
    {\ctx{\Gamma}{R} \lvar A}
    {\ctx{\Gamma}{R} \myrel A}
    \and
    \inferrule*[right=tm]
    {\ctx{\Gamma}{R} \myrel A}
    {\ctx{\Gamma}{R} \vdash A}
    \and
    \inferrule*[right=wk]
    {\ctx{\Gamma}{R}\phantom{, 0\Theta} \myrel A}
    {\ctx{\Gamma}{R}, 0\Theta \myrel A}
  \end{mathpar}
\end{frame}
\begin{frame}
  \frametitle{Traversal}
  \begin{theorem}
    \vspace{-1.5em}
    \begin{mathpar}
      \inferrule*[lab=trav]
      {\forall C.~(j : \Delta \ni C) \to
      (\basis j\rescomment\Psi)\Gamma \kitrel C
      \\ \resctx P \subres \resctx Q\rescomment\Psi
      \\ \ctx{\Delta}{Q} \vdash A}
      {\ctx{\Gamma}{P} \vdash A}
    \end{mathpar}
    \vspace{-2.5em}
  \end{theorem}
  \begin{proof}
  Let $\rho : \forall C.~(j : \Delta \ni C) \to
  (\basis j\rescomment\Psi)\Gamma \kitrel C$;
  $\resctx P \subres \resctx Q\rescomment\Psi$; and
  $M : \ctx{\Delta}{Q} \vdash A$.
  Induction on $M$:
  \begin{itemize}
    \item[\TirName{var}:]
      Have $j : \Delta \ni A$, $\resctx Q \subres \basis j$.
      Then, $\rho j : (\basis j\rescomment\Psi)\Gamma \kitrel A$, which can be
      coerced to $\Gamma(\resctx Q\rescomment\Psi) \kitrel A$, then
      $\ctx{\Gamma}{P} \kitrel A$.
      \TirName{tm} gives $\ctx{\Gamma}{P} \vdash A$.
    \item[\TirName{app}:]
      \newcommand\QM{\foo{Q}{M}}
      \newcommand\QN{\foo{Q}{N}}
      \newcommand\PM{\foo{P}{M}}
      \newcommand\PN{\foo{P}{N}}
      Have $M : \QM\Delta \vdash \fun{\rescomment rA}{B}$,
      $N : \QN\Delta \vdash A$,
      $\resctx Q \subres \QM + \rescomment r\QN$.
      Let $\PM := \QM\rescomment\Psi$, $\PN := \QN\rescomment\Psi$.
      Notice $\resctx P \subres \PM + \rescomment r\PN$.
      By induction, we get $\Gamma\PM \vdash \fun{\rescomment rA}{B}$ and
      $\Gamma\PN \vdash A$.
    \item[\TirName{lam}:]
      Have $M : \ctx{\Delta}{Q}, \ctxvar{x}{A}{r} \vdash B$.
      The bind lemma gives us $\rescomment{\Psi'}$ such that
      $\rowcat{\resctx P}{\rescomment r} \subres
      \rowcat{\resctx Q}{\rescomment r}\rescomment{\Psi'}$ and
      $\rho' : \forall C.~(j : \rowcat{\Delta}{A} \ni C) \to
      (\basis j\rescomment{\Psi'})\rowcat{\Gamma}{A} \kitrel C$.
      Then, by induction, we can get
      $\ctx{\Gamma}{P}, \ctxvar{x}{A}{r} \vdash B$, as required.
  \end{itemize}
  \end{proof}
\end{frame}
\begin{frame}
  \frametitle{The bind lemma}
  \begin{lemma}
    Given $\rescomment\Psi$ such that $\resctx P \subres \resctx Q\rescomment\Psi$
    and $\rho : \forall C.~(j : \Delta \ni C) \to
    (\basis j\rescomment\Psi)\Gamma \kitrel C$, we can find $\rescomment{\Psi'}$
    such that $\rowcat{\resctx P}{\resctx R} \subres
    \rowcat{\resctx Q}{\resctx R}\rescomment{\Psi'}$ and
    $\rho' : \forall C.~(j : \rowcat{\Delta}{\Theta} \ni C) \to
    (\basis j\rescomment{\Psi'})\rowcat{\Gamma}{\Theta} \kitrel C$.
  \end{lemma}
  \begin{proof}
    Let $\rescomment{\Psi'} := \rescomment\Psi \oplus I =
    \left(\begin{array}{c|c}
            \rescomment\Psi & 0 \\\hline 0 & I
          \end{array}\right)$.
    This satisfies the inequality.
    To construct $\rho'$, take cases on $j$:
    \begin{itemize}
      \item[$\inl j$]
        Have $j : \Delta \ni C$.
        Use $\rho$ to get $(\basis j\rescomment\Psi)\Gamma \kitrel C$, then
        \TirName{wk} to get
        $(\basis j\rescomment\Psi)\Gamma, 0\Theta \kitrel C$.
        By linear algebra, we can coerce this thingy as needed:
        $\basis{\inl j}\rescomment{\Psi'}
        = \rowcat{(\basis j)}{0}\rescomment{\Psi'}
        \subres \rowcat{(\basis j\rescomment{\Psi})}{0}$.
      \item[$\inr k$]
        Have $k : \Theta \ni C$, thus a new
        $\inr k : \rowcat{\Gamma}{\Theta} \ni C$.
        We have $\basis{\inr k}\rescomment{\Psi'}
        = \rowcat{0}{(\basis k)}\rescomment{\Psi'}
        \subres \rowcat{0}{(\basis k)}
        = \basis{\inr k}$, so we can get
        $(\basis{\inr k}\rescomment{\Psi'})\rowcat{\Gamma}{\Theta} \lvar C$,
        and by \TirName{vr}
        $(\basis{\inr k}\rescomment{\Psi'})\rowcat{\Gamma}{\Theta} \kitrel C$.
    \end{itemize}
  \end{proof}
\end{frame}
\begin{frame}
  \frametitle{Single substitution}
  \begin{corollary}
    \vspace{-1.5em}
    \begin{mathpar}
      \inferrule*[right=single-subst]
      {M : \ctx{\Gamma}{P}, \ctxvar{x}{A}{r} \vdash B
        \\ N : \ctx{\Gamma}{Q} \vdash A
        \\ \resctx R \subres \resctx P + \rescomment r\resctx Q}
      {M[N] : \ctx{\Gamma}{R} \vdash B}
    \end{mathpar}
    \vspace{-2em}
  \end{corollary}
  \begin{proof}
    We want to apply traversal at the following type:
    \begin{mathpar}
      \inferrule*[right=trav]
      {\forall C.~(j : \rowcat{\Gamma}{A} \ni C) \to
      (\basis j\rescomment\Psi)\Gamma \vdash C
      \\ \resctx R \subres \rowcat{\resctx P}{\rescomment r}\rescomment\Psi
      \\ \ctx{\Gamma}{P}, \ctxvar{x}{A}{r} \vdash B}
      {\ctx{\Gamma}{R} \vdash B}
    \end{mathpar}
    Let $\rescomment\Psi := [I, \resctx Q] = \colcat{I}{\resctx Q}$.
    Construct $\rho$ by cases:
    \begin{itemize}
      \item[$\inl i$]
        Have $i : \Gamma \ni C$. Notice that
        $\basis{\inl i}\rescomment\Psi \subres \basis i$, so \TirName{var} on
        $i$ suffices.
      \item[$\inr{\bullet}$]
        Have $A = C$. Notice that
        $\basis{\inr{\bullet}}\rescomment\Psi \subres \resctx Q$, so it suffices
        to give $N : \ctx{\Gamma}{Q} \vdash A$.
    \end{itemize}
  \end{proof}
\end{frame}
\begin{frame}
  \frametitle{Conclusion}
  \begin{block}{Contributions}
    \begin{itemize}
      \item Linear metatheory in an intuitionistic style
      \item A novel definition of simultaneous substitution
      \item Clarification of the (linear) algebra required
      \item Easy to formalise (as done in Agda:
        \url{https://github.com/laMudri/generic-lr})
    \end{itemize}
  \end{block}
  \begin{block}{Future work}
    \begin{itemize}
      \item Semiring homomorphisms
      \item Syntax-to-semantics traversals\footfullcite{ACMM17}
      \item A notion of good graded syntax\footfullcite{AACMM20}
      \item Partial/non-deterministic addition
    \end{itemize}
  \end{block}
\end{frame}
\end{document}
