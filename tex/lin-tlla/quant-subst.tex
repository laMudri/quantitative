\documentclass[submission,copyright,creativecommons]{eptcs}
\providecommand{\event}{SOS 2007} % Name of the event you are submitting to
\usepackage{breakurl}             % Not needed if you use pdflatex only.
\usepackage{underscore}           % Only needed if you use pdflatex.

\usepackage{stmaryrd}
\usepackage{mathpartir}
\usepackage{amssymb}
\usepackage{cmll}
\usepackage{xcolor}
\usepackage{paralist}
\usepackage{amsmath}
\usepackage{amsthm}
\usepackage{mathrsfs}
\usepackage{mathtools}
\usepackage{tabularx}
\usepackage{tikz-cd}
\usepackage[conor]{agda}

\DeclareFontFamily{U}{cal}{}
\DeclareFontShape{U}{cal}{m}{n}{<->cmsy10}{}
\DeclareSymbolFont{rcal}{U}{cal}{m}{n}
\DeclareSymbolFontAlphabet{\mathcal}{rcal}

\usepackage{thmtools}
\declaretheorem[numberwithin=section]{theorem}
\declaretheorem[numberlike=theorem]{conjecture}
\declaretheorem[numberlike=theorem]{proposition}
\declaretheorem[numberlike=theorem]{lemma}
\declaretheorem[numberlike=theorem]{corollary}
\declaretheorem[numberlike=theorem]{example}
\declaretheorem[numberlike=theorem]{definition}
\declaretheorem[numberlike=theorem]{remark}

\def\newelims{1}
\def\multnotation{1}
\ifx\newelims\undefined
  \def\newelims{0}
\fi
\ifx\multnotation\undefined
  \def\multnotation{0}
\fi

\newcommand{\fixme}[1]{{\color{red}\underline{\em{#1}}}}
\newcommand{\fixmeBA}[1]{{\color{red}{[Bob:{\em #1}]}}}


\newcommand{\bind}[2]{%
  \if\debruijn0%
  {#2}\{{#1}\}%
  \else%
  {#2}%
  \fi%
}
\newcommand{\ctx}[2]{%
  \if\multnotation0%
  #1^{\resctx{#2}}%
  \else%
  \resctx{#2}#1%
  \fi
}
\newcommand{\ctxvar}[3]{%
  \if\debruijn0%
  \if\multnotation0%
  #1 \stackrel{\rescomment{#3}}: #2%
  \else%
  \rescomment{#3}#1 : #2%
  \fi%
  \else%
  \rescomment{#3}#2%
  \fi%
}
\definecolor{res}{HTML}{008000}
\definecolor{resperm}{HTML}{008000}
\newcommand{\rescomment}[1]{{\color{res}#1}}
\newcommand{\rescommentperm}[1]{{\color{resperm}#1}}
\newcommand{\resctx}[1]{\rescomment{\mathcal{#1}}}
\newcommand{\resctxperm}[1]{\rescommentperm{\mathcal{#1}}}
\newcommand{\resmat}[1]{\rescomment{#1}}


\newcommand{\ann}[2]{#1 : #2}
\newcommand{\emb}[1]{\underline{#1}}

\newcommand{\base}[0]{\iota}

\newcommand{\fun}[2]{#1 \multimap #2}
\newcommand{\lam}[2]{%
  \if\debruijn0%
  \lambda #1.~#2%
  \else%
  \lambda #2%
  \fi%
}
\newcommand{\app}[2]{#1~#2}

\newcommand{\excl}[2]{\oc_{\rescomment{#1}} #2}
\newcommand{\bang}[1]{\left[#1\right]}
\newcommand{\bm}[4]{%
  \if\newelims0%
  \operatorname{bm}_{#1}(#2, \bind{#3}{#4})%
  \else%
  \if\debruijn0%
  \mathrm{let}~\bang{#3} = #2~\mathrm{in}~#4%
  \else%
  \mathrm{let}~\bang{-} = #2~\mathrm{in}~#4%
  \fi%
  \fi%
}

\newcommand{\tensorOne}[0]{1}
\newcommand{\unit}[0]{(\mathbin{_\otimes})}
\newcommand{\del}[3]{\if\newelims0%
\operatorname{del}_{#1}(#2, #3)%
\else%
\mathrm{let}~\unit = #2~\mathrm{in}~#3%
\fi}

\newcommand{\tensor}[2]{#1 \otimes #2}
\newcommand{\ten}[2]{(#1 \mathbin{_{\otimes}} #2)}
\newcommand{\prm}[5]{%
  \if\newelims0%
  \operatorname{pm}_{#1}(#2, \bind{#3, #4}{#5})%
  \else%
  \if\debruijn0%
  \mathrm{let}~\ten{#3}{#4} = #2~\mathrm{in}~#5%
  \else%
  \mathrm{let}~\ten{-}{-} = #2~\mathrm{in}~#5%
  \fi%
  \fi%
}

\newcommand{\withTOne}[0]{\top}
\newcommand{\eat}[0]{(\mathbin{_{\with}})}

\newcommand{\withT}[2]{#1 \with #2}
\newcommand{\wth}[2]{(#1 \mathbin{_\with} #2)}
\newcommand{\proj}[2]{\operatorname{proj}_{#1} #2}

\newcommand{\sumTZero}[0]{0}
\newcommand{\exf}[2]{\operatorname{ex-falso} #2}

\newcommand{\sumT}[2]{#1 \oplus #2}
\newcommand{\inj}[2]{\operatorname{inj}_{#1} #2}
\newcommand{\cse}[6]{%
  \if\newelims0%
  \operatorname{case}_{#1}(#2, \bind{#3}{#4}, \bind{#5}{#6})%
  \else%
  \if\debruijn0%
  \mathrm{case}~#2~\mathrm{of}~\inj{L}{#3} \mapsto #4
                            ~;~ \inj{R}{#5} \mapsto #6%
  \else%
  \mathrm{case}~#2~\mathrm{of}~\inj{L}{-} \mapsto #4 ~;~ \inj{R}{-} \mapsto #6
  \fi%
  \fi%
}

\newcommand{\listT}[1]{\operatorname{List} #1}
\newcommand{\nil}[0]{[]}
\newcommand{\cons}[2]{#1 :: #2}
\newcommand{\fold}[5]{\operatorname{fold} #1~#2~(#3,#4.~#5)}


\newcommand{\typed}[1]{\mathit{#1t}}
\newcommand{\resourced}[1]{\mathit{#1r}}


\newcommand{\lvar}{\mathrel{\mathrlap{\sqsupset}{\mathord{-}}}}
\newcommand{\sem}[1]{\left\llbracket #1 \right\rrbracket}

%\def\tobar{\mathrel{\mkern3mu  \vcenter{\hbox{$\scriptscriptstyle+$}}%
%                    \mkern-12mu{\to}}}
\newcommand\tobar{\mathrel{\ooalign{\hfil$\mapstochar\mkern5mu$\hfil\cr$\to$\cr}}}


\newcommand{\subres}{\trianglelefteq}
\newcommand{\subrctx}[2]{{#1} \subres {#2}}
\makeatletter
\newcommand{\subst}[2][]{\ext@arrow 0359\Rightarrowfill@{#1}{#2}}
\makeatother


\newenvironment{eqns}{\begin{array}{r@{\hspace{0.3em}}c@{\hspace{0.3em}}l}}{\end{array}}


\newcommand{\mat}[1]{\mathbf{#1}}
\newcommand{\vct}[1]{\mathbf{#1}}
\DeclarePairedDelimiter\basis{\langle}{\rvert}


\newcommand{\name}{\ensuremath{\lambda \resctxperm R}}


\DeclareMathOperator\kit{Kit}
\newcommand{\kitrel}{\mathbin{\blacklozenge}}


\DeclareMathOperator\id{id}
\DeclareMathOperator\inl{inl}
\DeclareMathOperator\inr{inr}
\DeclareMathOperator\Idx{Idx}


\newcommand{\zero}{\ensuremath{\rescomment 0}}
\newcommand{\linear}{\ensuremath{\rescomment 1}}
\newcommand{\unrestricted}{\ensuremath{\rescomment \omega}}
\newcommand{\instDILL}{\rescomment{01\omega}}

\newcommand{\unused}{\ensuremath{\rescomment{\mathit{unused}}}}
\newcommand{\true}{\ensuremath{\rescomment{\mathit{true}}}}
\newcommand{\valid}{\ensuremath{\rescomment{\mathit{valid}}}}
\newcommand{\instPD}{\ensuremath{\rescomment{\mathit{utv}}}}


\title{An Algebraically Structured Approach to Substructural Substitution}
\author{Robert Atkey
\institute{University of Strathclyde\\ Glasgow, United Kingdom}
\email{robert.atkey@strath.ac.uk}
\and
James Wood\thanks{James Wood is supported by an EPSRC Studentship.}
\institute{University of Strathclyde\\ Glasgow, United Kingdom}
\email{james.wood.100@strath.ac.uk}
}
\def\titlerunning{An Algebraically Structured Approach to Substructural
  Substitution}
\def\authorrunning{R. Atkey \& J. Wood}
\begin{document}
\maketitle

\begin{abstract}
  We present a nice way to prove syntactic lemmas for linear type theory.
\end{abstract}

\section{Introduction}

The basic metatheoretic results for typed $\lambda$-calculi, such as
preservation of typing under renaming, weakening, substitution and so
on, are crucial but quite boring to prove. In the case of calculi with
substructural typing disciplines and modalities, it can also be quite
easy to break these properties \cite{wadler91use}. It is desirable
therefore to use a proof assistant to prove these properties. This has
the double benefit of raising the level of confidence in the results,
and in focusing on the essential properties required to obtain them.

In this paper, we adapt the generic \emph{kits and traversals}
technique for proving admissibility of renaming and substitution due
to McBride \cite{rensub05} to a linear typed $\lambda$-calculus where
variables are annotated with values from a skew semiring denoting
those variables' \emph{usage} by terms. Our calculus, \name{}, is a
prototypical example of a linear ``quantitative'' or ``coeffect''
calculus in the style of
\cite{reed10distance,BrunelGMZ14,GhicaS14,PetricekOM14,Granule18}. By
selecting different semirings, we obtain systems equivalent to
Barber's Dual Intuitionistic Linear Logic \cite{Barber1996} or
Pfenning and Davis' S4 modal type theory \cite{judgmental}.

McBride's kits and traversals technique isolates properties required
to form binding respecting traversals of simply typed $\lambda$-terms,
so that renaming and substitution arise as specific
instantiations. Benton, Hur, Kennedy, and McBride \cite{bhkm12}
implement the technique in Coq and extend the technique to polymorphic
terms. Allais \emph{et al.}~\cite{AACMM20} generalise to a wider class
of syntax with binding and show that more general notions of
\emph{semantics} can be explored with the technique.

To adapt kits and traversals to linear usage annotated terms requires
us to not only respect the binding structure, but to also respect the
\emph{resource} structure. For instance, the resources associated with
a term being substituted in must be correctly distributed across all
the occurences of that term in the result. To aid us in tracking
resources correctly, we employ the linear algebra of vectors and
matrices induced by the skew semiring we are using. Usage annotations
on contexts are vectors, resource preserving maps of contexts are
matrices, and the linearity properties of the maps induced by matrices
are exactly the lemmas we need for showing that traversals (and hence
renaming, sub-usaging, and substitution) preserve typing.



% We build on previous work on coeffects
% \cite{BrunelGMZ14,GhicaS14,reed10distance,PetricekOM14} to produce a generic
% usage-constrained type system \name{}.
% This system first appeared in our earlier abstract, which explores the
% relational semantics of \name{}, giving concrete examples of free theorems we
% can obtain \cite{context-constrained}.
% In the process of formalising \name{}, we have been driven to find a clean
% statement and proof of substitution.

% Our definition of a substitution somewhat resembles the definition given in
% Petricek's thesis \cite[p. 137]{petricek-thesis}.
% However, our definition yields a simultaneous substitution in the style of
% Benton, Hur, Kennedy, and McBride \cite{bhkm12} by allowing for addition of
% annotations between substituting derivations.

The paper proceeds as follows.
In \autoref{sec:algebra}, we specify our requirements on the set of annotations
that will track usage of variables. A consequence of our formalisation is that we learn that we only need \emph{skew} semirings, a weaker structure than the partially ordered semirings usually used.
In \autoref{sec:syntax}, we use these annotations to define the system
\name{} in an intrinsically typed style.
Then, in \autoref{sec:metatheory}, we prove that \name{} admits renaming, sub-usaging, and
substitution by our extension of McBride's kits and traversals technique.
We conclude in \autoref{sec:conclusion} with some directions for future work.

The Agda formalisation of this work can be found at
\url{https://github.com/laMudri/generic-lr/tree/specific/src/Specific}.
% TODO: link to specific commit
It contains the definition, operators, and properties required of vectors and
matrices (approx. 790 lines) and the definition of \name{} and proofs of
renaming and substitution (approx. 530 lines).

\section{Skew semirings}\label{sec:algebra}

We shall use skew semirings where authors have previously used partially ordered
semirings (see, for example, the Granule definition of a \emph{usage algebra} \cite{Granule18}).
Elements of a skew semiring are used as \emph{usage annotations}, and describe
\emph{how} values are used in a program.
In the syntax for \name{}, each assumption will have a usage annotation,
describing how that assumption can be used in the derivation.
Addition describes how to combine multiple usages of an assumption, and
multiplication describes the action our graded $\oc$-modality can have.
The ordering describes the specificness of annotations.
If $\pi \subres \rho$, $\pi$ can be the annotation for a variable wherever
$\rho$ can be.
We can read this relation as ``$\textrm{supply} \subres \textrm{demand}$'' ---
where we demand that a variable be used according to $\rho$, it is also fine to
use it if it is annotated $\pi$.

Skew semirings are a generalisation of semirings, which are in turn a
generalisation of commutative semirings.
As such, readers unfamiliar with the more general structures may wish to think
in terms of the more specific structures.
Our formalisation was essential for noticing and sticking to the level of
generality we have done.

\begin{definition}
  A \emph{(left) skew monoid} is a structure $(\mathbf R, \subres, 1, *)$ such
  that $(\mathbf R, \subres)$ forms a partial order, $*$ is monotonic with
  respect to $\subres$, and the following laws hold.
  \begin{mathpar}
    1x \subres x
    \and x \subres x1
    \and (xy)z \subres x(yz)
  \end{mathpar}
\end{definition}

\begin{remark}
  A commutative skew monoid is just a commutative monoid.
\end{remark}

Skew-monoidal categories are due to Szlach\'anyi \cite{skew}, and the notion
introduced here of a skew monoid is a decategorification of the notion of
skew-monoidal category.

\begin{definition}
  A \emph{(left) skew semiring} is a structure
  $(\mathbf R, \subres, 0, +, 1, *)$ such that $(\mathbf R, \subres)$ forms a
  partial order, $+$ and $*$ are monotonic with respect to $\subres$,
  $(\mathbf R, 0, +)$ forms a commutative monoid, $(\mathbf R, \subres, 1, *)$
  forms a skew monoid, and we have the following distributivity laws.
  \begin{mathpar}
    0z \subres 0
    \and (x + y)z \subres xz + yz
    \and 0 \subres x0
    \and xy + xz \subres x(y + z)
  \end{mathpar}
\end{definition}

\begin{example}
  In light of the above remark, most ``skew'' semirings are actually
  just partially ordered semirings. An example that yields a system
  equivalent to DILL is the $0 \triangleright 1 \triangleleft \omega$
  semiring of ``unused'', ``linear'' and ``unrestricted''. See
  \cite{Granule18} for more examples.
\end{example}

We will only speak of \emph{left} skew semirings, and thus generally
omit the word ``left''.  A mnemonic for (left) skew semirings is
``multiplication respects operators on the left from left to right,
and respects operators on the right from right to left''.  One may
also describe multiplication as ``respecting'' and ``corespecting''
operators on the left and right, respectively.

From a skew semiring $\mathbf R$, we form finite vectors, which we
notate as $\mathbf R^n$, and matrices, which we notate as
$\mathbf{R}^{m\times n}$. In Agda, we represent vectors in
$\mathbf R^n$ as functions $\mathrm{Idx}~n \to \mathbf{R}$, where
$\mathrm{Idx}~n$ is the type of valid indexes in an $n$-tuple, and
matrices in $\mathbf{R}^{m\times n}$ as functions
$\mathrm{Idx}~m \to \mathrm{Idx}~n \to \mathbf{R}$.  Whereas elements
of $\mathbf R$ describe how individual \emph{variables} are used,
elements of $\mathbf R^n$ describe how all of the variables in an
$n$-length \emph{context} are used. We call such vectors \emph{usage
  contexts}, and take them to be row vectors. Matrices in
$\mathbf{R}^{m\times n}$ will be used to describe how usage contexts
are transformed by renaming and substitution in
\autoref{sec:metatheory}. We define $\subres$, $0$ and $+$ on vectors
and matrices pointwise. Basis vectors $\langle i \rvert$ (used to
represent usage contexts for individual variables), identity matrices
$\mat I$, matrix multiplication $*$, and matrix reindexing
${-}_{{-}\times{-}}$ are defined as follows:
\begin{mathpar}
  \begin{matrix*}[l]
    \langle {-} \rvert : \mathrm{Idx}~n \to \mathbf{R}^n \\
    \langle i \rvert_j :=
    \begin{cases}
      1, & \textrm{if }i = j \\
      0, & \textrm{otherwise} \\
    \end{cases}
  \end{matrix*}
  \and
  \begin{matrix*}[l]
    \mat I : \mathbf R^{m \times m} \\
    \mat I_{ij} := \langle i \rvert_j
  \end{matrix*}
  \and
  \begin{matrix*}[l]
    * : \mathbf R^{m \times n} \times \mathbf R^{n \times o} \to \mathbf R^{m \times o} \\
    (MN)_{ik} := \sum_j M_{ij}N_{jk}
  \end{matrix*}
  \and
  \begin{matrix*}[l]
    {-}_{{-}\times{-}} : \mathbf R^{m' \times n'}
    \times (\mathrm{Idx}~m \to \mathrm{Idx}~m')
    \times (\mathrm{Idx}~n \to \mathrm{Idx}~n')
    \to \mathbf R^{m \times n} \\
    \left(M_{f \times g}\right)_{i,j} := M_{f\,i,g\,j}
  \end{matrix*}
\end{mathpar}

We define vector-matrix multiplication by treating vectors
as$1$-height matrices.
  
% We make use of the following operators, which we also take to work on vectors
% (for example, we often multiply a row vector by a matrix, in that order).

\section{Syntax}\label{sec:syntax}

\begin{figure}[t]
  \caption{Typing rules of \name{}}
  \label{fig:rules}
  \begin{mathpar}
    \inferrule*[right=var]
    {x : \ctx{\Gamma}{R} \lvar A}
    {x : \ctx{\Gamma}{R} \vdash A}
  \end{mathpar}
  \begin{mathpar}
    \inferrule*[right=$\fun{}{}$-E]
    {M : \ctx{\Gamma}{P} \vdash \fun{A}{B}
      \\ N : \ctx{\Gamma}{Q} \vdash A
      \\ \resctx R \subres \resctx P + \resctx Q
    }
    {\app{M}{N} : \ctx{\Gamma}{R} \vdash B}
    \and
    \inferrule*[right=$\fun{}{}$-I]
    {\bind{x}M : \ctx{\Gamma}{R}, \ctxvar{x}{A}{1} \vdash B}
    {\lam{x}{\bind{x}M} : \ctx{\Gamma}{R} \vdash \fun{A}{B}}

    \and

    \inferrule*[right=$\tensorOne$-E]
    {M : \ctx{\Gamma}{P} \vdash \tensorOne{}
      \\ N : \ctx{\Gamma}{Q} \vdash C
      \\ \resctx R \subres \resctx P + \resctx Q
    }
    {\del{C}{M}{N} : \ctx{\Gamma}{R} \vdash C}
    \and
    \inferrule*[right=$\tensorOne$-I]
    {\resctx R \subres \rescomment{\vct 0}}
    {\unit{} : \ctx{\Gamma}{R} \vdash \tensorOne{}}
    \and
    \inferrule*[right=$\tensor{}{}$-E]
    {M : \ctx{\Gamma}{P} \vdash \tensor{A}{B}
      \\ \bind{x,y}N : \ctx{\Gamma}{Q}, \ctxvar{x}{A}{1}, \ctxvar{y}{B}{1}
      \vdash C
      \\ \resctx R \subres \resctx P + \resctx Q
    }
    {\prm{C}{M}{x}{y}{\bind{x,y}N} : \ctx{\Gamma}{R} \vdash C}
    \and
    \inferrule*[right=$\tensor{}{}$-I]
    {M : \ctx{\Gamma}{P} \vdash A
      \\ N : \ctx{\Gamma}{Q} \vdash B
      \\ \resctx R \subres \resctx P + \resctx Q
    }
    {\ten{M}{N} : \ctx{\Gamma}{R} \vdash \tensor{A}{B}}

    \and

    \inferrule*[right=$\sumTZero$-E]
    {M : \ctx{\Gamma}{P} \vdash \sumTZero{}
      \\ \resctx R \subres \resctx P + \resctx Q
    }
    {\exf{C}{M} : \ctx{\Gamma}{R} \vdash C}
    \and
    % (\textrm{no }\TirName{$\sumTZero$-I})
    % \and
    \inferrule*[right=$\sumT{}{}$-E]
    {M : \ctx{\Gamma}{P} \vdash \sumT{A}{B}
      \\ \bind{x}N : \ctx{\Gamma}{Q}, \ctxvar{x}{A}{1} \vdash C
      \\ \bind{y}O : \ctx{\Gamma}{Q}, \ctxvar{y}{B}{1} \vdash C
      \\ \resctx R \subres \resctx P + \resctx Q
    }
    {\cse{C}{M}{x}{\bind{x}N}{y}{\bind{y}O} : \ctx{\Gamma}{R} \vdash C}
    \and
    \inferrule*[right=$\sumT{}{}$-Il]
    {M : \ctx{\Gamma}{R} \vdash A}
    {\inj{L}{M} : \ctx{\Gamma}{R} \vdash \sumT{A}{B}}
    \and
    \inferrule*[right=$\sumT{}{}$-Ir]
    {M : \ctx{\Gamma}{R} \vdash B}
    {\inj{R}{M} : \ctx{\Gamma}{R} \vdash \sumT{A}{B}}

    \and

    % (\textrm{no }\TirName{$\withTOne$-E})
    % \and
    \inferrule*[right=$\withTOne$-I]
    { }
    {\eat{} : \ctx{\Gamma}{R} \vdash \withTOne}
    \and
    \inferrule*[right=$\withT{}{}$-El]
    {M : \ctx{\Gamma}{R} \vdash \withT{A}{B}}
    {\proj{L}{M} : \ctx{\Gamma}{R} \vdash A}
    \and
    \inferrule*[right=$\withT{}{}$-Er]
    {M : \ctx{\Gamma}{R} \vdash \withT{A}{B}}
    {\proj{R}{M} : \ctx{\Gamma}{R} \vdash B}
    \and
    \inferrule*[right=$\withT{}{}$-I]
    {M : \ctx{\Gamma}{R} \vdash A
      \\ N : \ctx{\Gamma}{R} \vdash B
    }
    {\wth{M}{N} : \ctx{\Gamma}{R} \vdash \withT{A}{B}}

    \and

    \inferrule*[right=$\excl{\rho}{}$-E]
    {M : \ctx{\Gamma}{P} \vdash \excl{\rho}{A}
      \\ \bind{x}N : \ctx{\Gamma}{Q}, \ctxvar{x}{A}{\rho} \vdash C
      \\ \resctx R \subres \resctx P + \resctx Q
    }
    {\bm{C}{M}{x}{\bind{x}N} : \ctx{\Gamma}{R} \vdash C}
    \and
    \inferrule*[right=$\excl{\rho}{}$-I]
    {M : \ctx{\Gamma}{P} \vdash A
      \\ \resctx R \subres \rescomment\rho\resctx P
    }
    {\bang{M} : \ctx{\Gamma}{R} \vdash \excl{\rho}{A}}
  \end{mathpar}
\end{figure}

We present the syntax of \name{} as an \emph{intrinsically} typed
syntax, as it is in our Agda formalisation. Intrinisic typing means
that we define well-typed terms as inhabitants of an inductive family
$\ctx\Gamma{R} \vdash A$ indexed by typing contexts $\Gamma$, usage
contexts $\resctx{R}$, and types $A$. Typing contexts are lists of
types. Usage contexts $\resctx{R}$ are vectors of elements of some
fixed skew semiring $\mathbf R$, with the same number of elements as
the typing context they are paired with. To highlight how usage
annotations are used in the syntax, we write all elements of
$\mathbf R$, and vectors and matrices thereof, in \rescomment{green}.

% This section corresponds to the Agda module \AgdaModule{Specific.Syntax}\footnote{link?}.

% \paragraph{Types}

The types of \name{} are given by the grammar:
\begin{displaymath}
  A,B,C ::= \iota \mid A \multimap B \mid 1 \mid A \otimes B \mid 0 \mid A \oplus B \mid \top \mid A \with B \mid \excl{\rho} A
\end{displaymath}
We have a base type $\iota$, function types $A \multimap B$, tensor
product types $A \otimes B$ with unit $1$, sum types $A \oplus B$ with
unit $0$, ``with'' product types $A \with B$ with unit $\top$, and an
exponential modality $\excl{\rho} A$ that is indexed by a usage
annotation $\rescomment{\rho}$.

We distinguish between \emph{plain} variables, values of type
$\Gamma \ni A$, which are indices into a context with a specified
type, and \emph{usage-checked} variables, values of type
$\ctx{\Gamma}{R} \lvar A$, which describe the conditions under which a
variable can be used in a context, subject to the constraints of usage
annotations. Formally, a usage-checked variable is a pair of a plain
variable $i : \Gamma \ni A$ and proof that
$\resctx R \subres \langle i \rvert$. The force of the latter
condition is that the selected variable $i$ must have a usage
annotation $\subres \rescomment 1$ in $\resctx{R}$, and all other
variables must have a usage annotation $\subres \rescomment 0$.

% \begin{definition}[plain variable, \AgdaRecord{IVar}]
%   We write $\Gamma \ni A$ as the type of variables in $\Gamma$ with type $A$.
% \end{definition}

% \begin{definition}[usage-checked variable, \AgdaRecord{LVar}]
%   We write $\ctx{\Gamma}{R} \lvar A$ as the type of $i : \Gamma \ni A$ such that
%   $\resctx R \subres \langle i \rvert$.
% \end{definition}

%\paragraph{Intrinsically Typed Terms}

The constructors for our intrinsically typed terms are presented in \autoref{fig:rules}.
In keeping with our intrinsic typing methodology, terms of \name{} are presented as constructors of the inductive family $\ctx\Gamma{R} \vdash A$, hence the notation $M : \ctx\Gamma{R} \vdash A$ instead of the more usual $\ctx\Gamma{R} \vdash M : A$. 
Our Agda formalisation uses de Bruijn indices to represent variables, but we have annotated the rules with variable names for ease of reading. 
% The corresponding Agda definition is \AgdaDatatype{Tm}.
Ignoring the \rescomment{usages}, the typing rules all look like their
simply typed counterparts; the only difference between the $\otimes$
and $\with$ products being their presentation in terms of pattern
matching and projections, respectively.
Thus the addition of usage contexts and constraints on them refines the usual simple typing to be usage constrained.
For instance, in the \TirName{$\otimes$-I} rule, the usage context $\rescomment R$ on the conclusion is constrained to be able to supply the sum $\rescomment P + \rescomment Q$ of the usage contexts of the premises.
As an example of using \name{}, if we instantiate $\mathbf R$ to be the $\{0, 1, \omega\}$ semiring, then we obtain a system that is equivalent to Barber's DILL \cite{Barber1996}.

% It is a refinement of simply typed $\lambda$-calculus sporting usage annotations
% from $\mathbf R$ on variables in the context and constraints on term formation
% based on these annotations.
% Type formers are mostly standard from intuitionistic linear logic --- functions,
% multiplicative and additive products and their units, and additive sums and
% their unit.
% Additionally, we have an annotated bang ($\excl{\rho}{A}$), forming a graded
% comonad (assuming definitions tweaked to account for $\subres$ and skewness).

% \section{Skew linear algebra}

% Elements and vectors of $\mathbf R$ suffice to define the syntax of
% \name{}, but to formally prove renaming and substitution admissible in
% \autoref{sec:metatheory}, we first need to develop some elementary
% skew linear algebra.

% \begin{definition}
%   A \emph{(left) skew semimodule} over a (left) skew semiring $\mathbf R$ is a
%   structure
%   $(\mathbf M, \subres,
%   0 : \mathbf M, + : \mathbf M \times \mathbf M \to \mathbf M,
%   * : \mathbf R \times \mathbf M \to \mathbf M)$ such that
%   \begin{itemize}
%     \item $+$ and $*$ are monotonic in both arguments.
%     \item $(\mathbf M, 0, +)$ is a commutative monoid
%     \item Scaling of a fixed element $m$ respects all of the skew semiring
%       structure; particularly:
%       \begin{mathpar}
%         0_\mathbf R m \subres 0_\mathbf M
%         \and
%         (\pi +_\mathbf R \rho) m \subres \pi m +_\mathbf M \rho m
%         \and
%         1_\mathbf R m \subres m
%         \and
%         (\pi *_\mathbf R \rho) m \subres \pi (\rho m)
%       \end{mathpar}
%     \item Scaling by a fixed element $\pi$ corespects the additive structure;
%       particularly:
%       \begin{mathpar}
%         0_\mathbf M \subres \pi 0_\mathbf M
%         \and
%         \pi m +_\mathbf M \pi n \subres \pi (m +_\mathbf M n)
%       \end{mathpar}
%   \end{itemize}
% \end{definition}

% \begin{definition}
%   A \emph{(left) skew semimodule homomorphism} or \emph{linear map} between
%   $\mathbf R$-left skew semimodules $\mathbf M$ and $\mathbf N$ is a monotone
%   function $T$ on the underlying sets written postfix such that all of the skew
%   semimodule structure is respected.
%   Particularly,
%   \begin{mathpar}
%     (0_\mathbf M)T \subres 0_\mathbf N
%     \and
%     (m +_\mathbf M n)T \subres (m)T +_\mathbf N (n)T
%     \and
%     (\pi *_\mathbf M m)T \subres \pi *_\mathbf N (m)T
%   \end{mathpar}
% \end{definition}

% The last axiom motivates the postfix notation.
% If we take application of $T$ to be a form of multiplication (as justified later
% in this section), then we do not want to require $T$ to commute with $\pi$.
% This remains important for non-skew semirings, but not commutative semirings.

% A concrete source of linear maps is matrices.

% \begin{lemma}\label{lem:mat-to-map}
%   Each matrix $M : \mathbf R^{m \times n}$ gives rise to a linear map
%   $T_M : \mathbf R^m \to \mathbf R^n$ by right-multiplication.
% \end{lemma}
% \begin{proof}
%   We observe the following inequalities.
%   Notice that in the last inequality, $M$ being on the right means that we do
%   not have to commute it past $\pi$.
%   To pull $\pi$ out of the sum, we use the fact that multiplication
%   \emph{corespects} the additive structure on the right.
%   \begin{itemize}
%     \item $\left(\vct 0M\right)_{k}
%       = \sum_j 0M_{jk}
%       \subres \sum_j 0
%       = 0
%       = \vct 0_{k}$
%     \item $\left((u + v)M\right)_{k}
%       = \sum_j \left(u_j + v_j\right)M_{jk}
%       \subres \sum_j \left(u_jM_{jk} + v_jM_{jk}\right)
%       = \sum_j u_jM_{jk} + \sum_j v_jM_{jk}
%       = (uM + vM)_{k}$
%     \item $\left((\pi u)M\right)_{k}
%       = \sum_j \left(\pi u_j\right)M_{jk}
%       \subres \sum_j \pi\left(u_jM_{jk}\right)
%       \subres \pi\sum_j u_jM_{jk}
%       = \left(\pi(uM)\right)_{k}$
%   \end{itemize}
% \end{proof}

% \begin{lemma}\label{lem:map-to-mat}
%   Each linear map $T : \mathbf R^m \to \mathbf R^n$ gives rise to a matrix
%   $M : \mathbf R^{m \times n}$ such that for each $u$, $uT \subres uM$.
% \end{lemma}
% \begin{proof}
%   Let $M_{jk} = (\langle j \rvert)T_k$.
%   Then, consider $uT_k$.
%   \[
%     uT_k \subres \left(\sum_j u_j\langle j \rvert\right)T_k
%     \subres \sum_j \left(u_j\langle j \rvert\right) T_k
%     \subres \sum_j u_j\left(\langle j \rvert T_k\right)
%     = \sum_j u_j M_{jk} = (uM)_k
%   \]
% \end{proof}

% \autoref{lem:map-to-mat} is peculiar to the skew setting; normally one would expect there to be a 1-1 relationship between linear maps and matrices.
% Intuitively, when multiplication is involved, equations are hard to come by, so
% we instead settle for inequalities.
% In what follows, matrices/linear maps will only appear to the right of a
% $\subres$, so when we ask for a matrix rather than a linear map, it is no loss
% of generality.

\section{Metatheory}\label{sec:metatheory}

McBride defines \emph{kits} \cite{rensub05,bhkm12}, which provide a general
method for giving admissible rules which are usually proven by induction on the
derivation.
To produce a kit, we give an indexed family
$\kitrel : \mathrm{Ctx} \times \mathrm{Ty} \to \mathrm{Set}$ and explain how to
inject variables, extract terms, and weaken by new variables coming from
binders.
In return, given a type-preserving map from variables in one context to
$\kitrel$-stuff in another (an \emph{environment}), we get a type-preserving
function between terms in these contexts.
Such a function is the intrinsic typing equivalent of an admissible rule.

To make the kit-based approach work in our usage-constrained setting, we make
modifications to both kits and environments.
Kits need not support arbitrary weakening, but only weakening by the
introduction of $\rescomment 0$-use variables.
The family $\kitrel$ must also respect $\subres$ of usage contexts.
Environments are equipped with a matrix mapping input usages to output usages.

We prove simultaneous substitution via renaming.
We take both renaming and substitution as corollaries of the \emph{traversal}
principle (\autoref{thm:trav}) yielded from kits and environments.

The definitions of \emph{kit} and \emph{environment}, as well as the proof of
\hyperref[thm:trav]{traversal}, all can be found in the module
\AgdaModule{Specific.Syntax.Traversal}.

% Plan:
% - Just define a kit, saying what it is used for (i.e. sketch the traversal we get)
% - State that 'vr', 'tm' and 'wk' are also in Conor's version.
% - Note that we are being careful with resources in the 'wk' bit
% - 'psh' is new, to cope with sub-usaging

\subsection{Kits, Environments, and Traversals}

A kit is a structure on $\vdash$-like relations $\kitrel$, intuitively
giving a way in which $\kitrel$ lives between the usage-checked variable
judgement $\lvar$ and the typing judgement $\vdash$.
The components $\mathit{vr}$ and $\mathit{tm}$ are basically unchanged from
McBride's original kits.
The component $\mathit{wk}$ only differs in that new variables are given
annotation $\rescomment 0$, which intuitively marks them as weakenable.
The requirement $\mathit{psh}$ is new, and allows us to fix up usage contexts
via skew algebraic reasoning.

\begin{definition}[\AgdaRecord{Kit}]\label{def:kit}
  For any $\kitrel : \mathrm{Ctx} \times \mathrm{Ty} \to \mathrm{Set}$, let
  $\kit(\kitrel)$ denote the type of \emph{kits}.
  A kit comprises the following functions for all $\resctx P$, $\resctx Q$,
  $\Gamma$, $\Delta$, and $A$.
  \begin{mathpar}
    \mathit{psh} : \resctx P \subres \resctx Q \to
    \ctx{\Gamma}{Q} \kitrel A \to \ctx{\Gamma}{P} \kitrel A
    \and
    \mathit{vr} : \ctx{\Gamma}{P} \lvar A \to
    \ctx{\Gamma}{P} \kitrel A
    \\
    \mathit{tm} : \ctx{\Gamma}{P} \kitrel A \to
    \ctx{\Gamma}{P} \vdash A
    \and
    \mathit{wk} : \ctx{\Gamma}{P} \kitrel A \to
    \ctx{\Gamma}{P}, \ctx{\Delta}{\vct 0} \kitrel A
  \end{mathpar}
\end{definition}

An inhabitant of $\ctx{\Gamma}{P} \kitrel A$ is described as
\emph{stuff in $\ctx{\Gamma}{P}$ of type $A$}.

\paragraph{Environments} In simple intuitionistic type theory, an
environment is just a type-preserving function from variables in the
old context $\Delta$ to stuff in the new context $\Gamma$.  That is,
the environment is an inhabitant of
$\Delta \ni A \to \Gamma \kitrel A$.  The traversal function
$\mathit{trav}$ turns such an environment into a map between terms,
$\Delta \vdash A \to \Gamma \vdash A$.

In \name{}, we want inhabitants of
$\ctx{\Delta}{Q} \vdash A \to \ctx{\Gamma}{P} \vdash A$.
We can see that an environment of type
$\ctx{\Delta}{Q} \lvar A \to \ctx{\Gamma}{P} \kitrel A$ would
be insufficient --- $\ctx{\Delta}{Q} \lvar A$ can only be inhabited when
$\resctx Q$ is compatible with a basis vector, so our environment would be
trivial in more general cases.
Instead, we care about non-usage-checked variables $\Delta \ni A$.

Our understanding of an environment is that it should simultaneously map all of
the usage-checked variables in $\ctx{\Delta}{Q}$ to stuff in $\ctx{\Gamma}{P}$
in a way that preserves usage.
As such, we want to map each variable $j : \Delta \ni A$ not to $A$-stuff in
$\ctx{\Gamma}{P}$, but rather $A$-stuff in $\resctx P_j\Gamma$,
where $\resctx P_j$ is some fragment of $\resctx P$.
Precisely, when weighted by $\resctx Q\lvert j \rangle$, we want these
$\resctx P_j$ to sum to $\resctx P$, so as to provide ``enough'' usage to cover
all of the variables $j$.
When we collect all of the $\resctx P_j$ into a matrix $\rescomment\Psi$, we
notice that the condition just described is stated succinctly via a
vector-matrix multiplication $\resctx Q\rescomment\Psi$.

This culminates to give us the following requirements.

\begin{definition}[\AgdaRecord{Env}]\label{def:env}
  For any $\kitrel$, $\resctx P$, $\resctx Q$, $\Gamma$, and $\Delta$,
  where $\Gamma$ and $\Delta$ have lengths $m$ and $n$ respectively,
  let $\ctx{\Gamma}{P} \subst{\kitrel} \ctx{\Delta}{Q}$ denote the
  type of \emph{environments}.  An environment comprises a pair of a
  matrix $\rescomment\Psi : \mathbf R^{n \times m}$ and a mapping of
  variables
  $\mathit{act} : (j : (\Delta \ni A)) \to (\langle j
  \rvert\rescomment\Psi)\Gamma \kitrel A$, such that
  $\resctx P \subres \resctx Q \rescomment\Psi$.
% \begin{itemize}
%   \item $\rescomment\Psi : \mathbf R^{n \times m}$
%   \item $\mathit{act} :
%     (j : (\Delta \ni A)) \to (\langle j \rvert\rescomment\Psi)\Gamma \kitrel A$
%   \item with usage condition $\resctx P \subres \resctx Q \rescomment\Psi$.
% \end{itemize}
\end{definition}

Our main result is the following, which we will instantiate to prove
admissibility of renaming (\autoref{cor:ren}), sub-usaging
(\autoref{cor:subusage}), and substitution (\autoref{cor:sub}). The
proof is in \autoref{sec:proof-of-traversal}.

\newcommand{\thmtrav}{%
  Given a kit on $\kitrel$ and an environment
  $\ctx{\Gamma}{P} \subst{\kitrel} \ctx{\Delta}{Q}$, we get a function
  $\ctx{\Delta}{Q} \vdash A \to \ctx{\Gamma}{P} \vdash A$.%
}
\begin{theorem}[traversal, \AgdaFunction{trav}]\label{thm:trav}
  \thmtrav
\end{theorem}

\subsection{Renaming}

We now show how to use traversals to prove that renaming and
sub-usaging are admissible. This subsection corresponds to the Agda
module \AgdaModule{Specific.Syntax.Renaming}.

\begin{definition}[\AgdaFunction{LVar-kit}]\label{def:lvar-kit}
  Let $\lvar\textrm{-kit} : \kit(\lvar)$ be defined with the following
  fields.
  \begin{description}
    \item[$\mathit{psh}~(\mathit{PQ} : \resctx P \subres \resctx Q)
      : \ctx{\Gamma}{Q} \lvar A \to \ctx{\Gamma}{P} \lvar A$:]
      The only occurrence of the usage context $\resctx Q$ in the definition of
      $\lvar$ is to the left of a $\subres$.
      Applying transitivity in this place gets us the required term.
    \item[$\mathit{vr} : \ctx{\Gamma}{P} \lvar A \to \ctx{\Gamma}{P} \lvar A
      := \mathrm{id}$].
    \item[$\mathit{tm} : \ctx{\Gamma}{P} \lvar A \to \ctx{\Gamma}{P} \vdash A
      := \TirName{var}$].
    \item[$\mathit{wk} : \ctx{\Gamma}{P} \lvar A
      \to \ctx{\Gamma}{P}, \ctx{\Delta}{\vct 0} \lvar A$:]
      A basis vector extended by $\rescomment 0$s is still a basis
      vector: if that we have $\resctx P \subres \langle i \rvert$ for some $i$,
      we also have
      $\resctx P, \rescomment{\vct 0} \subres \langle \inl i \rvert$.
  \end{description}
\end{definition}

Environments for renamings are special in that the matrix $\rescomment\Psi$ can
be calculated from the action of the renaming on non-usage-checked variables.

\begin{lemma}[\AgdaFunction{ren-env}]\label{lem:ren-env}
  Given a type-preserving mapping of plain variables
  $f : \Delta \ni A \to \Gamma \ni A$ such that
  $\resctx P \subres \resctx Q\rescomment I_{f\times\id}$,
  we can produce a $\lvar$-environment of type
  $\ctx{\Gamma}{P} \subst{\lvar} \ctx{\Delta}{Q}$.
\end{lemma}
\begin{proof}
  The environment has $\rescomment\Psi := \rescomment I_{f\times\id}$,
  so the usage condition holds by assumption.
  Now, $\mathit{act}$ is required to have type
  $(j : \Delta \ni A) \to (\langle j \rvert\rescomment\Psi)\Gamma \lvar A$.
  Take arbitrary $j : \Delta \ni A$.
  Then, we have $f~j : \Gamma \ni A$, so all that is left is to show that $f~j$
  forms a usage-checked variable of type
  $(\langle j \rvert\rescomment\Psi)\Gamma \lvar A$.
  This amounts to proving
  $\langle j \rvert\rescomment\Psi \subres \langle f~j \rvert$.
%  We prove this pointwise.
  Let $i : \Gamma \ni A$, then we have
  % Then, we have the following.
  % \[
    $(\langle j \rvert\rescomment\Psi)_i
    \subres \rescomment\Psi_{j,i}
    = \rescomment I_{f\,j,i}
    = \langle f~j \rvert_i$.
%  \]
\end{proof}

\begin{corollary}[renaming, \AgdaFunction{ren}]\label{cor:ren}
  Given a type-preserving mapping of plain variables
  $f : \Delta \ni A \to \Gamma \ni A$ such that
  $\resctx P \subres \resctx Q\rescomment I_{f\times\id}$,
  we can produce a function of type
  $\ctx{\Delta}{Q} \vdash A \to \ctx{\Gamma}{P} \vdash A$.
\end{corollary}
% \begin{proof}
%   By the lemmas from this subsection and \autoref{thm:trav}.
% \end{proof}

\begin{corollary}[sub-usaging, \AgdaFunction{subuse}]\label{cor:subusage}
  Given $\resctx P \subres \resctx Q$, then we have a function
  $\ctx{\Gamma}{Q} \vdash A \to \ctx{\Gamma}{P} \vdash A$.
\end{corollary}

\subsection{Substitution}

Now that we have renaming, we can use it with traversals to prove that
simultaneous well-usaged substitution is admissible. This subsection
corresponds to the Agda module
\AgdaModule{Specific.Syntax.Substitution}.

\begin{definition}[\AgdaFunction{Tm-kit}]\label{tm-kit}
  Let $\vdash\textrm{-kit} : \kit(\vdash)$ be defined with the following
  fields.
  \begin{description}
    \item[$\mathit{psh}~(\mathit{PQ} : \resctx P \subres \resctx Q)
      : \ctx{\Gamma}{Q} \vdash A \to \ctx{\Gamma}{P} \vdash A$:]
      This is \hyperref[cor:subusage]{Corollary \ref*{cor:subusage} (sub-usaging)}.
      
      % We can use a specialised version of \hyperref[cor:ren]{renaming} in which
      % the variable mapping $f$ is the identity function.
      % It remains to check that
      % $\resctx P \subres \resctx Q\rescomment I_{\id\times\id}$, which is
      % obvious from the assumption $\mathit{PQ}$.
      % This is the argument used in the Agda module
      % \AgdaModule{Specific.Syntax.Subuse}, definition \AgdaFunction{subuse}.

      % Alternatively, we can note that in every rule for
      % $\ctx{\Gamma}{R} \vdash A$, $\resctx R$ only ever appears to the left of a
      % $\subres$, so $\ctx{\Gamma}{R} \vdash A$ must be contravariant in
      % $\resctx R$.
    \item[$\mathit{vr} : \ctx{\Gamma}{P} \lvar A \to \ctx{\Gamma}{P} \vdash A
      := \TirName{var}$].
    \item[$\mathit{tm} : \ctx{\Gamma}{P} \vdash A \to \ctx{\Gamma}{P} \vdash A
      := \mathrm{id}$].
    \item[$\mathit{wk} : \ctx{\Gamma}{P} \vdash A \to \ctx{\Gamma}{P},
      \ctx{\Delta}{\vct 0} \vdash A$:] We use \hyperref[cor:ren]{Corollary \ref*{cor:ren} (renaming)}, with
      $f : \Gamma \ni A \to \Gamma, \Delta \ni A$ being the embedding
      $\inl$.  It remains to check that
      $(\resctx P, \rescomment{\vct 0}) \subres \resctx P\rescomment
      I_{\inl\times\id}$.  We prove this pointwise.  Let
      $i : \Gamma, \Delta \ni A$, and take cases on whether $i$ is
      from $\Gamma$ or from $\Delta$.  If $i = \inl i'$ for an
      $i' : \Gamma \ni A$, we must show that
      $\resctx P_{i'} \subres (\resctx P\rescomment
      I_{\inl\times\id})_{\inl i'}$.  But we have the following.
      \[
      \resctx P_{i'} \subres (\resctx P\rescomment I)_{i'}
      = \sum_{j : \Gamma \ni A} \resctx P_j\rescomment I_{j,i'}
      = \sum_{j : \Gamma \ni A} \resctx P_j\rescomment I_{\inl j,\inl i'}
      = (\resctx P\rescomment I_{\inl\times\id})_{\inl i'}.
      \]
      If $i = \inr i'$ for an $i' : \Delta \ni A$, we must show that
      $\rescomment 0 \subres
      (\resctx P\rescomment I_{\inl\times\id})_{\inr i'}$.
      But we have the following.
      \[
      \rescomment 0 \subres (\resctx P\rescomment{\vct 0})_{i'}
      = \sum_{j : \Gamma \ni A} \resctx P_j\rescomment{\vct 0}_{j,i'}
      = \sum_{j : \Gamma \ni A} \resctx P_j\rescomment I_{\inl j,\inr i'}
      = (\resctx P\rescomment I_{\inl\times\id})_{\inr i'}.
      \]
  \end{description}
\end{definition}

\begin{corollary}[substitution, \AgdaFunction{sub}]\label{cor:sub}
  Given an environment of type
  $\ctx{\Gamma}{P} \subst{\vdash} \ctx{\Delta}{Q}$ (i.e., a
  well-usaged simultaneous substitution), we get a function of type
  $\ctx{\Delta}{Q} \vdash A \to \ctx{\Gamma}{P} \vdash A$.
\end{corollary}
% \begin{proof}
%   By \autoref{thm:trav} using \hyperref[tm-kit]{$\vdash\textrm{-kit}$}.
% \end{proof}

\subsection{Proof of traversal}
\label{sec:proof-of-traversal}

The proof of the traversal theorem follows the same structure as in
McBride's article, extended with proof obligations to show that we are
correctly respecting the usage annotations. We must first prove a
lemma that shows that environments can be pushed under binders.

\begin{lemma}[bind, \AgdaFunction{bind}]\label{lem:bind}
  Given a kit on $\kitrel$, we can extend an environment of type
  $\ctx{\Gamma}{P} \subst{\kitrel} \ctx{\Delta}{Q}$, to an environment of type
  $\ctx{\Gamma}{P}, \ctx{\Theta}{R} \subst{\kitrel}
  \ctx{\Delta}{Q}, \ctx{\Theta}{R}$.
\end{lemma}
\begin{proof}
  Let the environment we are given be
  $(\rescomment\Psi : \mathbf R^{n \times m},
  \mathit{act} : (j : \Delta \ni A) \to (\langle j \rvert\rescomment\Psi)\Gamma \kitrel A)$,
  with $\resctx P \subres \resctx Q \rescomment\Psi$.
  We are trying to construct
  $(\rescomment{\Psi'} : \mathbf R^{(n + o) \times (m + o)},
  \mathit{act'} : (j : \Delta, \Theta \ni A) \to
  (\langle j \rvert\rescomment{\Psi'})(\Gamma, \Theta) \kitrel A)$,
  with $\resctx P, \resctx R \subres (\resctx Q, \resctx R) \rescomment{\Psi'}$.
%
  Let \(
    \rescomment{\Psi'} := \left(\begin{array}{c|c}
                                  \rescomment\Psi & \rescomment{\mat 0}
                                  \\ \hline
                                  \rescomment{\mat 0} & \rescomment{\mat I}
                                \end{array}\right).
  \)
  With this definition, our required condition splits into the easily checked
  conditions
  $\resctx P \subres \resctx Q\rescomment\Psi + \resctx R\rescomment{\mat 0}$
  and
  $\resctx R \subres
  \resctx Q\rescomment{\mat 0} + \resctx R\rescomment{\mat I}$.
%
  For $\mathit{act'}$, we take cases on whether $j$ is from $\Delta$ or from
  $\Theta$.
%
  In the $\Delta$ case, $\mathit{act}$ gets us an inhabitant of
  $(\langle j \rvert\rescomment\Psi)\Gamma \kitrel A$.
  Notice that
  $\langle j \rvert\rescomment{\Psi'} =
  \langle j \rvert\rescomment\Psi, \rescomment{\vct 0}$,
  so we want to get from $(\langle j \rvert\rescomment\Psi)\Gamma \kitrel A$ to
  $(\langle j \rvert\rescomment\Psi)\Gamma, \rescomment{\vct 0}\Theta
  \kitrel A$.
  We can get this using $\mathit{wk}$ from our kit.
%
  In the $\Theta$ case, notice that
  $\langle j \rvert\rescomment{\Psi'} = \rescomment{\vct 0}, \langle j \rvert$.
  In other words, $\langle j \rvert\rescomment{\Psi'}$ is a basis vector, so we
  actually have usage-checked
  $(\langle j \rvert\rescomment{\Psi'})(\Gamma, \Theta) \lvar A$.
  Thus, we can use $\mathit{vr}$ from our kit to get
  $(\langle j \rvert\rescomment{\Psi'})(\Gamma, \Theta) \kitrel A$, as required.
\end{proof}

\newtheorem*{thm:trav}{\autoref{thm:trav}}
\begin{thm:trav}[traversal, \AgdaFunction{trav}]
  \thmtrav
\end{thm:trav}
\begin{proof}
  By induction on the syntax of $M$. In the \TirName{var} $x$ case,
  where $x : \ctx{\Delta}{Q} \lvar A$: By definition of $\lvar$, we
  have that $\resctx Q \subres \langle j \rvert$ for some $j$.
  Applying the action of the environment, we have
  $(\langle j \rvert\rescomment\Psi)\Gamma \kitrel A$.  We then have
  $\resctx P \subres \resctx Q\rescomment\Psi \subres \langle j
  \rvert\rescomment\Psi$, so using the fact that stuff appropriately
  respects subusaging ($\mathit{psh}$), we have
  $\ctx{\Gamma}{P} \kitrel A$.  Finally, using $\mathit{tm}$, we get a
  term $\ctx{\Gamma}{P} \vdash A$, as required.

  Non-\TirName{var} cases are generally handled in the following way.
  If the input usage context $\resctx Q$ is split up into a linear
  combination of zero or more usage contexts $\resctx Q_{i}$, obtain a
  similar splitting of $\resctx P$ by setting
  $\resctx P_{i} := \resctx Q_{i}\rescomment\Psi$.  This works out
  because of the linearity of matrix multiplication (in particular,
  multiplication respects operations on the left). This yields
  environments of type
  $\resctx P_{i}\Gamma \subst{\kitrel} \resctx Q_{i}\Delta$ for the
  subterms to use with the inductive hypothesis. If any subterms bind
  variables, apply \autoref{lem:bind} as appropriate.
  % \begin{description}
  %   \item[\TirName{var} $x$, where $x : \ctx{\Delta}{Q} \lvar A$:]
  %     By definition of $\lvar$, we have that
  %     $\resctx Q \subres \langle j \rvert$ for some $j$.
  %     Applying the action of the environment, we have
  %     $(\langle j \rvert\rescomment\Psi)\Gamma \kitrel A$.
  %     We then have
  %     $\resctx P \subres \resctx Q\rescomment\Psi
  %     \subres \langle j \rvert\rescomment\Psi$,
  %     so using the fact that stuff appropriately respects subusaging
  %     ($\mathit{psh}$), we have $\ctx{\Gamma}{P} \kitrel A$.
  %     Finally, using $\mathit{tm}$, we get a term $\ctx{\Gamma}{P} \vdash A$, as
  %     required.
    % \item[\TirName{$\withTOne$-I} $\eat$]
    %   We want to produce a term $\ctx{\Gamma}{P} \vdash \withTOne$, which is
    %   trivial with the rule \TirName{$\withTOne$-I}.
    % \item[\TirName{$\withT{}{}$-I} $\wth{M}{N}$, where
    %   $M : \ctx{\Delta}{Q} \vdash A$, $N : \ctx{\Delta}{Q} \vdash B$:]
    %   Inductively, we get $M' : \ctx{\Gamma}{P} \vdash A$ and
    %   $N' : \ctx{\Gamma}{P} \vdash B$ from
    %   $M$ and $N$, respectively.
    %   We can combine $M'$ and $N'$ with \TirName{$\withT{}{}$-I} to get
    %   $\wth{M'}{N'} : \ctx{\Gamma}{P} \vdash \withT{A}{B}$, as required.
    % \item[\TirName{$\withT{}{}$-El}, \TirName{$\withT{}{}$-Er},
    %   \TirName{$\sumT{}{}$-Il}, \TirName{$\sumT{}{}$-Ir}:]
    %   These all follow simply like the last two cases.
    % \item[\TirName{$\tensorOne$-I} $\unit$, where
    %   $\resctx Q \subres \rescomment{\vct 0}$:]
    %   We want to use \TirName{$\tensorOne$-I} to conclude
    %   $\unit : \ctx{\Gamma}{P}$.
    %   To do this, we must show that $\resctx P \subres \rescomment{\vct 0}$.
    %   From the environment, we know that
    %   $\resctx P \subres \resctx Q\rescomment\Psi$.
    %   But because $\resctx Q \subres \rescomment{\vct 0}$, we get
    %   $\resctx P \subres \rescomment{\vct 0}\rescomment\Psi
    %   \subres \rescomment{\vct 0}$, as required.
    % \item[\TirName{$\tensor{}{}$-I} $\ten{M}{N}$, where
    %   $M : \ctx{\Delta}{Q_{\mathnormal M}} \vdash A$,
    %   $N : \ctx{\Delta}{Q_{\mathnormal N}} \vdash B$,
    %   $\resctx Q \subres \resctx Q_{\rescomment M} + \resctx Q_{\rescomment N}$:]
    %   We want to inductively traverse $M$ and $N$.
    %   This will produce terms
    %   $M' : \ctx{\Gamma}{P_{\mathnormal M}} \vdash A$ and
    %   $N' : \ctx{\Gamma}{P_{\mathnormal N}} \vdash B$, for some choice of
    %   $\resctx P_{\rescomment M}$ and $\resctx P_{\rescomment N}$.
    %   Furthermore, we want
    %   $\resctx P \subres \resctx P_{\rescomment M} + \resctx P_{\rescomment N}$,
    %   so that we can combine $M'$ and $N'$ using \TirName{$\tensor{}{}$-I},
    %   getting $\ten{M'}{N'} : \ctx{\Gamma}{P} \vdash \tensor{A}{B}$, as
    %   required.
%
    %   Let
    %   $\resctx P_{\rescomment M} := \resctx Q_{\rescomment M}\rescomment\Psi$
    %   and
    %   $\resctx P_{\rescomment N} := \resctx Q_{\rescomment N}\rescomment\Psi$,
    %   where $\rescomment\Psi$ comes from the environment.
    %   We must make environments of types
    %   $\ctx{\Gamma}{P_{\mathnormal M}}
    %   \subst{\kitrel} \ctx{\Delta}{Q_{\mathnormal M}}$
    %   and
    %   $\ctx{\Gamma}{P_{\mathnormal N}}
    %   \subst{\kitrel} \ctx{\Delta}{Q_{\mathnormal N}}$.
    %   Notice that the data in these can be carried over from the environment we
    %   were given, but we must recheck the usage condition in each case.
    %   However, both of these just come down to the reflexivity of $\subres$.
%
    %   Finally, in order to apply \TirName{$\tensor{}{}$-I}, we must check that
    %   $\resctx P \subres \resctx P_{\rescomment M} + \resctx P_{\rescomment N}$.
    %   This follows by:
    %   \[
    %   \resctx P
    %   \subres \resctx Q\rescomment\Psi
    %   \subres (\resctx Q_{\rescomment M} + \resctx Q_{\rescomment N})
    %   \rescomment\Psi
    %   \subres \resctx Q_{\rescomment M}\rescomment\Psi +
    %   \resctx Q_{\rescomment N}\rescomment\Psi
    %   = \resctx P_{\rescomment M} + \resctx P_{\rescomment N}.
    %   \]
    % \item[\TirName{$\fun{}{}$-E}, \TirName{$\tensorOne{}$-E},
    %   \TirName{$\sumTZero$-E}:]
    %   These all follow similarly to the above case (ignoring the lack of a
    %   subterm $N$ in the \TirName{$\sumTZero$-E} case).
    % \item[\TirName{$\excl{\rho}{}$-I} $\bang{M}$, where
    %   $M : \ctx{\Delta}{Q_{\mathnormal M}} \vdash A$,
    %   $\resctx Q \subres \rescomment\rho \resctx Q_{\rescomment M}$:]
    %   Following a similar strategy to the case above, let
    %   $\resctx P_{\rescomment M} := \resctx Q_{\rescomment M}\rescomment\Psi$.
    %   If we get a term $M' : \ctx{\Gamma}{P_{\mathnormal M}} \vdash A$, we will
    %   be able to apply \TirName{$\excl{\rho}{}$-I} to get the result because
    %   \[
    %   \resctx P_{\rescomment M} = \resctx Q_{\rescomment M}\rescomment\Psi
    %   \subres (\rescomment\rho\resctx Q)\rescomment\Psi
    %   \subres \rescomment\rho(\resctx Q\rescomment\Psi)
    %   \subres \rescomment\rho\resctx P.
    %   \]
%
    %   To get $M'$, we traverse $M$ with the same environment at type
    %   $\ctx{\Gamma}{P_{\mathnormal M}}
    %   \subst{\kitrel} \ctx{\Delta}{Q_{\mathnormal M}}$.
    %   The usage condition reduces to the definition of
    %   $\resctx P_{\rescomment M}$.
    % \item[\TirName{$\fun{}{}$-I} $\lam{x}{M}$, where
    %   $M : \ctx{\Delta}{Q}, \ctxvar{x}{A}{1} \vdash B$:]
    %   We want to inductively traverse $M$, yielding a term
    %   $M' : \ctx{\Gamma}{P}, \ctxvar{x}{A}{1}$, to which we apply
    %   \TirName{$\fun{}{}$-I} to get the desired result.
    %   The environment we use to traverse $M$ is the result of the
    %   \hyperref[lem:bind]{bind lemma} applied to the environment we were given.
    % \item[\TirName{$\sumT{}{}$-E}, where
    %   $M : \ctx{\Delta}{Q_{\mathnormal M}} \vdash \sumT{A}{B}$,
    %   $N : \ctx{\Delta}{Q_{\mathnormal N}}, \ctxvar{x}{A}{1} \vdash C$,
    %   $O : \ctx{\Delta}{Q_{\mathnormal N}}, \ctxvar{y}{B}{1} \vdash C$,
    %   $\resctx Q \subres
    %   \resctx Q_{\rescomment M} + \resctx Q_{\rescomment N}$:]
    %   This is the most complex rule, combining splitting, sharing, and binding.
    %   Let
    %   $\resctx P_{\rescomment M} := \resctx Q_{\rescomment M}\rescomment\Psi$
    %   and
    %   $\resctx P_{\rescomment N} := \resctx Q_{\rescomment N}\rescomment\Psi$.
    %   In order to traverse $M$, we repurpose the environment at type
    %   $\ctx{\Gamma}{P_{\mathnormal M}}
    %   \subst{\kitrel} \ctx{\Delta}{Q_{\mathnormal M}}$.
    %   This amounts to checking that
    %   $\resctx P_{\rescomment M} \subres
    %   \resctx Q_{\rescomment M}\rescomment\Psi$, which is obvious.
%
    %   In order to traverse $N$ and $O$, we do similar to produce an environment
    %   at type
    %   $\ctx{\Gamma}{P_{\mathnormal N}}
    %   \subst{\kitrel} \ctx{\Delta}{Q_{\mathnormal N}}$.
    %   Then, we use the \hyperref[lem:bind]{bind lemma} to extend this
    %   (separately) to environments of type
    %   $\ctx{\Gamma}{P_{\mathnormal N}}, \ctxvar{x}{A}{1}
    %   \subst{\kitrel} \ctx{\Delta}{Q_{\mathnormal N}}, \ctxvar{x}{A}{1}$ and
    %   $\ctx{\Gamma}{P_{\mathnormal N}}, \ctxvar{y}{B}{1}
    %   \subst{\kitrel} \ctx{\Delta}{Q_{\mathnormal N}}, \ctxvar{y}{B}{1}$.
    %   These are enough to traverse $N$ and $O$, respectively.
%
    %   Finally, we apply \TirName{$\sumT{}{}$-E} to all of the traversed
    %   derivations.
    % \item[\TirName{$\tensor{}{}$-E}, \TirName{$\excl{\rho}{}$-E}:]
    %   These both feature a similar combination of splitting and binding to the
    %   previous case.
    %   The \TirName{$\excl{\rho}{}$-E} rule involves binding a variable with
    %   arbitrary annotation $\rescomment\rho$, but notice that the
    %   \hyperref[lem:bind]{bind lemma} handles this perfectly well.
%  \end{description}
\end{proof}

\section{Conclusion}\label{sec:conclusion}

We have extended McBride's method of kits and traversals to proving
admissibility of renaming, sub-usaging and substitution to the linear
usage-annotated calculus \name{}. In the course of doing so, we have
discovered that only skew semirings are required, and the importance
of linear algebra for stating and proving these results.

% In this paper, we have presented \name{}, a simply typed calculus with usage
% constraints via annotations on variables.
% For it, we have extended McBride's method of kits \cite{rensub05}.



Our work is similar in spirit to the work of Licata, Shulman, and
Riley \cite{LicataSR17}, which gives a proof of cut elimination for a
large class of substructural single-conclusion sequent calculi.  The
class of natural deduction systems we consider here is likely less
general, but not directly comparable. We leave a complete comparison
to future work. They have not formalised their work.
% In particular, we focus on ``at least linear'' systems, and prove an inherently
% linear result.

Mechanisation of the metatheory of substructural $\lambda$-calculi has
not recieved the same level of attention as intuitionistic
typing. ``Straightforward'' translations from paper presentations to
formal presentations are difficult, due to incompatibilies between the
standard de Bruijn representation of binding and the splitting of
contexts. Allais' \emph{Typing with Leftovers}
\cite{allais:LIPIcs:2018:10049} overcame this by tracking usage in an
I/O style. Rouvoet et al.~\cite{RPKV20} and Crary \cite{crary10} offer
other formalisations in Agda and Twelf respectively. Usage
annotations, arising from the coeffect literature, in tandem with the
linear algebraic approach we have presented here offer an approach
that, in our experience, is amenable to formalisation.

We plan to generalise the generic syntax with binding framework of
Allais \emph{et al.}~\cite{AACMM20} to work with usage annotations,
which would allow us to prove admissible rules for an even wider class
of substructural calculi.

\bibliographystyle{eptcsalpha}
\bibliography{../quantitative}
\end{document}
