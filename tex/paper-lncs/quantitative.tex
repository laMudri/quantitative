% This is samplepaper.tex, a sample chapter demonstrating the
% LLNCS macro package for Springer Computer Science proceedings;
% Version 2.20 of 2017/10/04
%
\documentclass[runningheads]{llncs}
%
\usepackage{graphicx}
% Used for displaying a sample figure. If possible, figure files should
% be included in EPS format.

\usepackage{stmaryrd}
\usepackage{mathpartir}
\usepackage{amssymb}
\usepackage{cmll}
\usepackage{xcolor}
\usepackage{paralist}
\usepackage{amsmath}
\usepackage{mathrsfs}
\usepackage{mathtools}
\usepackage{multirow}
\usepackage{relsize}
\usepackage{tabularx}
\usepackage{tikz-cd}
\usepackage{breqn}

\usetikzlibrary{arrows}

\usepackage{hyperref}
\renewcommand\UrlFont{\color{blue}\rmfamily}

\def\newelims{1}
\def\multnotation{1}
\ifx\newelims\undefined
  \def\newelims{0}
\fi

\newcommand{\bind}[2]{\{#1\}#2}
%\newcommand{\ctx}[2]{{#1}^{\rescomment{#2}}}
\newcommand{\ctx}[2]{{#2}^{#1}}
\newcommand{\ctxvar}[3]{#1 \stackrel{\rescomment{#3}}: #2}
%\newcommand{\rescomment}[1]{{\color{red}#1}}
\newcommand{\rescomment}[1]{{#1}}


\newcommand{\ann}[2]{#1 : #2}
\newcommand{\emb}[1]{\underline{#1}}

\newcommand{\base}[0]{\iota}

\newcommand{\fun}[2]{#1 \multimap #2}
\newcommand{\lam}[2]{\lambda #1.~#2}
\newcommand{\app}[2]{#1~#2}

\newcommand{\excl}[2]{\oc_{#1} #2}
\newcommand{\bang}[1]{\operatorname{bang} #1}
\newcommand{\bm}[4]{\if\newelims0%
\operatorname{bm}_{#1}(#2, \bind{#3}{#4})%
\else%
\mathrm{let}~\bang{#3} = #2~\mathrm{in}~#4 : #1%
\fi}

\newcommand{\tensorOne}[0]{1}
\newcommand{\unit}[0]{{*}_\otimes}
\newcommand{\del}[3]{\if\newelims0%
\operatorname{del}_{#1}(#2, #3)%
\else%
\mathrm{let}~\unit = #2~\mathrm{in}~#3 : #1%
\fi}

\newcommand{\tensor}[2]{#1 \otimes #2}
\newcommand{\ten}[2]{(#1, #2)_{\otimes}}
\newcommand{\prm}[5]{\if\newelims0%
\operatorname{pm}_{#1}(#2, \bind{#3, #4}{#5})%
\else%
\mathrm{let}~\ten{#3}{#4} = #2~\mathrm{in}~#5 : #1%
\fi}

\newcommand{\withTOne}[0]{\top}
\newcommand{\eat}[0]{{*}_{\with}}

\newcommand{\withT}[2]{#1 \with #2}
\newcommand{\wth}[2]{(#1, #2)_{\with}}
\newcommand{\proj}[2]{\operatorname{proj}_{#1} #2}

\newcommand{\sumTZero}[0]{0}
\newcommand{\exf}[2]{\operatorname{ex-falso}_{#1}(#2)}

\newcommand{\sumT}[2]{#1 \oplus #2}
\newcommand{\inj}[2]{\operatorname{inj}_{#1} #2}
\newcommand{\cse}[6]{\if\newelims0%
\operatorname{case}_{#1}(#2, \bind{#3}{#4}, \bind{#5}{#6})%
\else%
\mathrm{case}~#2~\mathrm{of}~\inj{0}{#3} \mapsto #4; \inj{1}{#5} \mapsto #6%
\fi}


\newcommand{\typed}[1]{\mathit{#1t}}
\newcommand{\resourced}[1]{\mathit{#1r}}


\newcommand{\sem}[1]{\llbracket #1 \rrbracket}

%\def\tobar{\mathrel{\mkern3mu  \vcenter{\hbox{$\scriptscriptstyle+$}}%
%                    \mkern-12mu{\to}}}
\newcommand\tobar{\mathrel{\ooalign{\hfil$\mapstochar\mkern5mu$\hfil\cr$\to$\cr}}}


\newcommand{\lemref}[1]{\hyperref[#1]{Lemma \ref*{#1}}}
\newcommand{\exref}[1]{\hyperref[#1]{Example \ref*{#1}}}
\newcommand{\judgeref}[1]{\hyperref[#1]{Judgement \ref*{#1}}}
\newcommand{\propref}[1]{\hyperref[#1]{Proposition \ref*{#1}}}
\newcommand{\diagref}[1]{\hyperref[#1]{Diagram \ref*{#1}}}
\newcommand{\thmref}[1]{\hyperref[#1]{Theorem \ref*{#1}}}
\newcommand{\defref}[1]{\hyperref[#1]{Definition \ref*{#1}}}

\begin{document}
%
\title{A Relational Semantics Tracking Usage %Capturing Usage Restrictions
  \thanks{James Wood is supported by an EPSRC Studentship.}}
%
%\titlerunning{Abbreviated paper title}
% If the paper title is too long for the running head, you can set
% an abbreviated paper title here
%
\author{Robert Atkey\inst{1}\orcidID{0000-1111-2222-3333} \and
  James Wood\inst{1}\orcidID{1111-2222-3333-4444}}
%
\authorrunning{R. Atkey, J. Wood}
% First names are abbreviated in the running head.
% If there are more than two authors, 'et al.' is used.
%
\institute{University of Strathclyde, Glasgow, G1 1XQ, United Kingdom
\email{\{robert.atkey,james.wood.100\}@strath.ac.uk}\\
\url{https://www.strath.ac.uk/}}
%
\maketitle              % typeset the header of the contribution
%
\begin{abstract}
The abstract should briefly summarize the contents of the paper in
150--250 words.

\keywords{First keyword  \and Second keyword \and Another keyword.}
\end{abstract}

\section{Introduction}
\label{sec:introduction}
When verifying functional programs in a dependently typed programming language,
we often end up writing a program twice --- once to implement the program we
want to run, and once to prove a simple property of the program.
A standard example of this is proving that a sorting function is a permutation.
Below we have the $\textrm{sort}$ function, defined idiomatically by iteration,
and a conventional proof $\textrm{sort-perm}$ showing that it is a permutation.
We assume an $\textrm{insert}$ function, and proof $\textrm{insert-perm} :
\forall x,\mathit{xs}.~\textrm{insert}~x~xs \simeq x :: xs$.

\[
  \begin{array}{ll}
    \begin{array}[t]{l}
      \textrm{sort} : \textrm{List}~A \to \textrm{List}~A \\
      \textrm{sort} = \textrm{foldr}~[]~\textrm{insert}
    \end{array}
    \quad & \quad
      \begin{array}[t]{l}
        \textrm{sort-perm} : \forall\mathit{xs}.~\textrm{sort}~\mathit{xs} \simeq \mathit{xs} \\
        \textrm{sort-perm}~[] = [] \\
        \textrm{sort-perm}~(x :: \mathit{xs}) = \\
        \quad \textrm{insert-perm}~x~(\textrm{sort}~\mathit{xs}) \\
        \quad {}\bullet (x :: \textrm{sort-perm}~\mathit{xs})
      \end{array}
  \end{array}
\]

This proof is slightly tricky, but it is proving something obvious.
The $\textrm{insert}$ function never drops or duplicates its inputs into the
output, so it is obviously a permutation, and $\textrm{sort}$ is only made up of
functions that do no dropping or duplicating, so is also a permutation.
The aim of this paper is to make observations like this formal.

\fixme{Old stuff} In normal typed $\lambda$-calculi, variables may be used multiple
times, in multiple contexts, for multiple reasons, as long as the
types agree. The disciplines of linear types \cite{girard87linear} and
coeffects \cite{PetricekOM14,BrunelGMZ14,GhicaS14} refine this by
tracking variable usage. We might track how many times a variable is
used, or if it is used co-, contra-, or invariantly. Such a discipline
yields a general framework of ``context constrained computing'', where
constraints on variables in the context tell us something interesting
about the computation being performed.
% Thus we put the
% type information to work to tell us facts about programs that might
% not otherwise be apparent.

We will present work in progress on capturing the ``intensional''
properties of programs via a family of Kripke indexed relational
semantics that refines a simple set-theoretic semantics of
programs. The value of our approach lies in its generality. We can
accommodate the following examples:
\begin{enumerate}
\item Linear types that capture properties like ``all list
  manipulating programs are permutations''. This example uses the
  Kripke-indexing to track the collection of datums currently being
  manipulated by the program.
\item Monotonicity coeffects that track whether a program uses inputs
  co-, contra-, or in-variantly (or not at all).
\item Sensitivity typing, tracking the sensitivity of programs in
  terms of input changes. This forms the core of systems for
  differential privacy \cite{reed10distance}.
\item Information flow typing, in the style of the Dependency Core
  Calculus \cite{abadi99core}.
\end{enumerate}

The syntax and semantics we present here have been formalised in Agda:
\url{https://github.com/laMudri/quantitative/}.

Our main contributions are:

\begin{itemize}
\item A rigorous statement of substitution for a substructural type system
\item A Kripke-indexed relational semantics providing strong free theorems
\item A formalisation of this work in Agda
\end{itemize}

\subsection{Related work}

We follow closely and extend the approaches of Petricek, Orchard, and Mycroft,
and Ghica and Smith \cite{PetricekOM14,GhicaS14}.
In particular, our framework is generic in a partially ordered semiring of
\emph{usage constraints}, which are placed on each variable in the context.
This is distinct from the approach taken by Brunel et al and the Granule project
\cite{BrunelGMZ14,Granule18}, where \emph{unannotated} variables also exist, and
are treated linearly.
Both of the latter systems contain a \emph{dereliction} rule, stating that a
variable annotated $\rescomment 1$ can be coerced to an unannotated variable.
This rule can be justified by the graded comonad unit law
$\fun{\excl{1}{A}}{A}$, so adds no new axioms to be met when we want to produce
semantics.
However, the distinction between annotated and unannotated variables does cause
complexity in the syntax.
For one thing, the dereliction rule is not syntax-directed.
For another, the operation of merging contexts is now partial, and the scaling
operation is not uniform or also partial.

Beyond previous work that assumed all variables were annotated, we allow a full
complement of propositional intuitionistic linear logic connectives.
Particularly, we have tensor products, with products, sums, and the bang
modality, as opposed to just functions.
In contrast to this previous work, our functions are not annotated with a usage
constraint; we prefer a combination of the function arrow and the bang modality
so as to improve modularity of the connectives.

Our main novel language feature compared to previous work on usage-constrained
typing is an account of an inductive type (tensor lists) in both syntax and
semantics.
We believe that this account could be generalised to cover all strictly positive
inductive types in a relatively straightforward manner.

\section{Motivation}
\label{sec:motivation}
In this section, we give a brief informal overview of the properties we wish to
obtain of programs for free by restricting usage.
We use this to motivate the design of syntax and semantics detailed in
\autoref{sec:syntax} and \autoref{sec:semantics} respectively.

\subsection{Syntax}

Our fundamental syntactic principle is that we will restrict use of the variable
rule by encoding in contexts \emph{how} its variables can be used.
To each variable in the context, we attach a \emph{usage annotation}.
We may use a given variable only if it can be used in a plain manner, and all
other variables in that context can be unused (discarded).

Unlike types, we should expect the usage annotations of variables to change in a
typing derivation.
For example, suppose we want a linear $\lambda$-calculus, where each
$\lambda$-bound variable has to be used exactly once.
Then in this language, we want to write a curried function of two arguments that
pairs those two arguments together.
Na\"ively, we can write
$\lambda x.\lambda y.~(x, y) : \fun{A}{\fun{B}{\tensor{A}{B}}}$.
When we check this, we have to use $x$ and discard $y$ in the left of the pair,
and use $y$ and discard $x$ in the right.
Doing both of these must constitute using both $x$ and $y$, eventually
discarding neither, implying some notion of accumulation of usages.

To deal with this formally, we can set the usage annotations to be the natural
numbers, with the intention of counting how many times a variable is used.
We can discard a variable annotated $0$, and we can plainly use variables
annotated $1$.
So, to use $x$ in our example, we must be trying to conclude
$\ctxvar{x}{A}{1}, \ctxvar{y}{B}{0} \vdash x : A$, and to use $y$, the
annotations must be the other way round.
Then, forming a pair lets us pointwise add together usage annotations, giving
conclusion
$\ctxvar{x}{A}{1+0}, \ctxvar{y}{B}{0+1} \vdash (x, y) : \tensor{A}{B}$.
In general, we require the set of usage annotations to have an addition
operator, as well as its unit $0$ and designated ``plain use'' annotation $1$.

We also want a way to reify the idea of a variable usable in any particular way
into a value in its own right.
For example, we may want a value that can be used exactly twice, rather than
exactly once.
For this, we introduce the \emph{graded bang} type constructor $\excl{\rho}$,
where $\rho$ is a usage annotation.
This has value constructor $\bang{-}$, and using pattern matching notation
allows us to write $\lambda\bang{x}.~(x, x) : \fun{\excl{2}{A}}{\tensor{A}{A}}$.
Before pattern matching, we have a variable $\ctxvar{b}{\excl{2}{A}}{1}$, and
after pattern matching, it is used up (has annotation $0$) and we get a new
variable $\ctxvar{x}{A}{2}$.

To produce an open term that can be used twice is to produce an open term that
can be used once, but in a context where each variable can be used twice as many
times as it was in producing the term once.
Formally, if
$\ctxvar{x_1}{A_1}{\pi_1}, \ldots, \ctxvar{x_n}{A_n}{\pi_n} \vdash t : B$, then
$\ctxvar{x_1}{A_1}{\rho*\pi_1}, \ldots, \ctxvar{x_n}{A_n}{\rho*\pi_n} \vdash
\bang t : \excl{\rho}{B}$.
This means that we have multiplication on usage annotations, of which $1$ is the
unit.

\subsection{Semantics}

The point of constraining the use of variables is to restrict ourselves to
certain classes of well behaved terms.
For example, we may be interested in the linear terms, or the monotonic terms,
or the terms that are not too sensitive to change.
We say that the terms that use variables in accordance with the rules given by
the syntax and usage annotations are \emph{well provisioned}.
We provide a unified denotational semantics as a tool to show that any well
provisioned term really has the properties we wanted of it.

We start by giving terms a standard $\mathrm{Set}$ semantics, written $\sem
A$, $\sem \Gamma$, and $\sem t : \sem \Gamma \to \sem A$ for types, contexts and
terms, respectively.
When written in a dependently typed programming language, this is an interpreter
that takes a metalanguage value for each variable in the context and produces a
metalanguage value as a result.
If we have a term in context with one variable, we can consider whether $\sem t
: \sem A \to \sem B$ is monotonic.
This would mean that if a bigger $\sem A$ value is given, the resulting $\sem B$
value will also be bigger.
To capture this, we interpret each type $A$ as a binary endorelation over $\sem
A$, written $\sem{A}^R \subseteq \sem A \times \sem A$.
Then, given our one-variable term $\ctxvar{x}{A}{1} \vdash t : B$, its semantics
says that it preserves the relation.
Explicitly, $\sem{t}^R :
\forall a, a'.~\sem{A}^R(a, a') \implies \sem{B}^R(\sem t~a, \sem t~a')$.
% The property we want is the following square, where $\Gamma \vdash t : A$,
% $\gamma, \gamma' \in \sem\Gamma$, $a, a' \in \sem A$.

% \begin{tikzcd}
%   \gamma \arrow[r] \arrow[d, "\sem t" left] & \gamma' \arrow[d, "\sem t"] \\
%   a \arrow[r] & a'
% \end{tikzcd}

For other properties, however, we also need the relation to be preserved
relative to a \emph{world}.
For example, in sensitivity analysis, we want say that a perturbation of
$\varepsilon$ in the environment leads to at most a perturbation of
$\varepsilon$ in the value produced.
The world is this $\varepsilon$.
\fixme{Too early to mention $\otimes$ vs $\&$?}
Additionally, we want the perturbation of an environment to be the sum of
perturbations of its variables, meaning that we need to be able to add worlds
together.
Finally, to work out the perturbation in a variable $\ctxvar{x}{A}{\rho}$, we
must consider the usage annotation $\rho$.
For sensitivity analysis, these annotations are going to be scaling factors, so
the action of an annotation $\rho$ on a world $\varepsilon$ will produce the
world $\rho\varepsilon$.
In general, usage annotations will be interpreted as actions on worlds.


\section{Syntax}
\label{sec:syntax}
\subsection{Typing}
\begin{mathpar}
  \inferrule
  {(x : T) \in \Gamma}
  {\Gamma \vdash x \in T}

  \and

  \inferrule
  {\Gamma \vdash e \in S \\ S <: T}
  {\Gamma \vdash T \ni \emb{e}}
  \and
  \inferrule
  {\Gamma \vdash S \ni s}
  {\Gamma \vdash \ann{s}{S} \in S}

  \and

  \inferrule
  {\Gamma, x : S \vdash T \ni t}
  {\Gamma \vdash \fun{S}{T} \ni \lam{x}{t}}
  \and
  \inferrule
  {\Gamma \vdash f \in \fun{S}{T} \\ \Gamma \vdash S \ni s}
  {\Gamma \vdash \app{f}{s} \in T}

  \inferrule
  {\Gamma \vdash S \ni s}
  {\Gamma \vdash \excl{\rho}{S} \ni \bang{s}}
  \and
  \inferrule
  {\Gamma \vdash e \in \excl{\rho}{S} \\ \Gamma, x : S \vdash T \ni t}
  {\Gamma \vdash T \ni \bm{T}{e}{x}{t}}

  \and

  \inferrule
  { }
  {\Gamma \vdash \tensorOne \ni \unit}
  \and
  \inferrule
  {\Gamma \vdash e \in \tensorOne \\ \Gamma \vdash S \ni s}
  {\Gamma \vdash \del{S}{e}{s} \in S}

  \and

  \inferrule
  {\Gamma \vdash S \ni s \\ \Gamma \vdash T \ni t}
  {\Gamma \vdash \tensor{S}{T} \ni \ten{s}{t}}
  \and
  \inferrule
  {\Gamma \vdash e \in \tensor{S}{T} \\ \Gamma, x : S, y : T \vdash U \ni u}
  {\Gamma \vdash \prm{U}{e}{x}{y}{u} \in U}

  \and

  \inferrule
  { }
  {\Gamma \vdash \withTOne \ni \eat}
  \and
  \text{(no $\withTOne$ elim)}

  \and

  \inferrule
  {\Gamma \vdash S \ni s \\ \Gamma \vdash T \ni t}
  {\Gamma \vdash \withT{S}{T} \ni \wth{s}{t}}
  \and
  \inferrule
  {\Gamma \vdash e \in \withT{S_0}{S_1}}
  {\Gamma \vdash \proj{i}{e} \in S_i}

  \text{(no $\sumTZero$ intro)}
  \and
  \inferrule
  {\Gamma \vdash e \in \sumTZero}
  {\Gamma \vdash \exf{S}{e} \in S}

  \inferrule
  {\Gamma \vdash S_i \ni s}
  {\Gamma \vdash \sumT{S_0}{S_1} \ni \inj{i}{s}}
  \and
  \inferrule
  {\Gamma \vdash e \in \sumT{S}{T}
    \\ \Gamma, x : S \vdash U \ni s \\ \Gamma, y : T \vdash U \ni t}
  {\Gamma \vdash \cse{U}{e}{x}{s}{y}{t} \in U}
\end{mathpar}

\subsection{Resourcing}


\section{Semantics}
\label{sec:semantics}
\paragraph{Underlying Semantics} We give a standard semantics of types
and well typed terms into sets and functions. This semantics ignores
the usage information. For types, we have:
\begin{displaymath}
  \begin{array}{ll}
    \llbracket \base \rrbracket = A_\base \\
    \llbracket \fun{S}{T} \rrbracket = \llbracket S \rrbracket \rightarrow \llbracket T \rrbracket &
    \llbracket \excl{\rho}{S} \rrbracket = \llbracket S \rrbracket \\
    \llbracket \tensorOne \rrbracket = \llbracket \withTOne \rrbracket = \{*\} &
    \llbracket \tensor{S}{T} \rrbracket = \llbracket \withT{S}{T} \rrbracket = \llbracket S \rrbracket \times \llbracket T \rrbracket \\
    \llbracket \sumTZero \rrbracket = \{\} &
    \llbracket \sumT{S}{T} \rrbracket = \llbracket S \rrbracket \uplus \llbracket T \rrbracket \\
  \end{array}
\end{displaymath}
Contexts are interpreted as left-nested products. Terms are assigned
the usual semantics as functions $\sem{t} : \sem{\Gamma} \to \sem{S}$.

\paragraph{Usage-tracking semantics} To derive interesting properties
from our type system, we refine the set-theoretic semantics by
Kripke-indexed binary relations. This gives a fundamental lemma for
our system, that when instantitated in different ways captures the
examples in the introduction.

Our framework is parameterised by a category $\mathcal{W}$ of
\emph{possible worlds} that track how resources are distributed by
programs. To interpret resource separation, we assume that
$\mathcal{W}$ has \emph{symmetric promonoidal structure}: profunctors
$J : 1 \tobar \mathcal{W}$ and
$P : \mathcal{W} \times \mathcal{W} \tobar \mathcal{W}$ such
that $P \odot (J \times 1) \cong 1$, $P \odot (1 \times J) \cong 1$,
$P \odot (1 \times P) \cong P \odot (P \times 1)$, and
$P \cong P \odot (\pi_2 \times \pi_1)$, and the triangle, pentagon,
and hexagon laws hold\footnote{We don't need the laws to hold to prove
  the fundamental lemma.}.

We now assign to each type $T$ a functor
$\sem{T}^R : \mathcal{W}^{op} \to \mathrm{Rel}~\sem{T}$ that captures
a notion of $\mathcal{W}$-indexed ``indistinguishability''. To
interpret $\oc_\rho S$, we assume we are given a relation transformer
$\oc_A : R^{op} \to \mathrm{Rel}(A)^{\mathcal{W}^{op}} \to
\mathrm{Rel}(A)^{\mathcal{W}^{op}}$
that satisfies the axioms of a monoidal exponential comonad. The
interesting cases are for functions, $\otimes$-products and the
$\oc_\rho$ modality:
\begin{displaymath}
  \begin{array}{l}
    \llbracket \fun{S}{T} \rrbracket^R~w~(f,f') = \\
    \quad \forall x,y.~P(y,w)x \Rightarrow \forall s,s'.~\llbracket S \rrbracket^R y (s, s') \Rightarrow \llbracket
    T \rrbracket^R x (f~s, f'~s')
    \\
    \llbracket \tensor{S}{T} \rrbracket^R~w~((s, t), (s', t')) =\\
    \quad
    \exists x,y.~P(x,y)w \wedge \llbracket S \rrbracket^R x (s, s') \wedge
    \llbracket T \rrbracket^R y (t, t')
    \\
    \llbracket \excl{\rho}{S} \rrbracket^R~w~(s,s')=
    \oc_\rho \llbracket S \rrbracket^R~w~(s,s')\\
  \end{array}
\end{displaymath}
Contexts
$\ctxvar{x_1}{S_1}{\rho_1}, \ldots, \ctxvar{x_n}{S_n}{\rho_n}$ are
interpreted as if they were
$\sem{(\cdots(1 \otimes \oc_{\rho_1}S_1) \cdots \otimes \oc_{\rho_n}
  S_n)}$.
With these definitions, we can prove the following fundamental lemma
for our Kripke-indexed relational semantics.

% $R \times S$ and $R \uplus S$ are defined pointwise on relations.
% Particularly, there are two cases for $R \uplus S$:

% \begin{itemize}
%   \item $R(r, r')$ implies $(R \uplus S)(\mathrm{inl}~r, \mathrm{inl}~r')$.
%   \item $S(s, s')$ implies $(R \uplus S)(\mathrm{inr}~s, \mathrm{inr}~s')$.
% \end{itemize}

% We assume a family of natural transformations $\oc$ satisfying the following laws.

%   \begin{mathpar}
%     \rho \leq \pi \implies (\oc_\pi R~w \implies \oc_\rho R~w) \and
%     \oc_0 R~w \implies J w \and
%     \oc_{\rho+\pi} R~w~(a, b) \implies \exists x,y. P(x,y)w \wedge \oc_\rho R~x~a \wedge \oc_\pi R~y~b \and
%     \oc_1 R~w \iff R~w \and
%     \oc_{\rho \cdot \pi} R~w \iff \oc_\rho(\oc_\pi R)~w
%   \end{mathpar}

% The semantics of a context $\ctxvar{x_1}{S_1}{\rho_1}, \ldots, \ctxvar{x_n}{S_n}{\rho_n}$ is given by $\tensor{\llbracket \excl{\rho_1}{S_1}}{\tensor{\ldots}{\excl{\rho_n}{S_n}}} \rrbracket^R$.

% This indexed relational semantics gives us a family of logical relations.
% The fundamental lemma is as follows.

\begin{theorem}[Fundamental Lemma]
  \begin{displaymath}
    \ctx{\Gamma}{\Delta} \vdash t : T \implies \llbracket \ctx{\Gamma}{\Delta} \rrbracket^R w~(\gamma, \gamma') \implies \llbracket T \rrbracket^R w~(\sem{t}\gamma, \sem{t}\gamma')
  \end{displaymath}
\end{theorem}


% Local Variables:
% TeX-master: "quantitative"
% End:


\section{Instantiations}
\label{sec:instantiations}
\input{instantiations}

\section{Conclusions}
\label{sec:future-work}
\input{future-work}
%
% ---- Bibliography ----
%
% BibTeX users should specify bibliography style 'splncs04'.
% References will then be sorted and formatted in the correct style.

\bibliographystyle{splncs04}
\bibliography{quantitative}

\end{document}
