In normal typed $\lambda$-calculi, variables may be used multiple
times, in multiple contexts, for multiple reasons, as long as the
types agree. The disciplines of linear types \cite{girard87linear} and
coeffects \cite{PetricekOM14,BrunelGMZ14,GhicaS14} refine this by
tracking variable usage. We might track how many times a variable is
used, or if it is used co-, contra-, or invariantly. Such a discipline
yields a general framework of ``context constrained computing'', where
constraints on variables in the context tell us something interesting
about the computation being performed.
% Thus we put the
% type information to work to tell us facts about programs that might
% not otherwise be apparent.

We will present work in progress on capturing the ``intensional''
properties of programs via a family of Kripke indexed relational
semantics that refines a simple set-theoretic semantics of
programs. The value of our approach lies in its generality. We can
accommodate the following examples:
\begin{enumerate}
\item Linear types that capture properties like ``all list
  manipulating programs are permutations''. This example uses the
  Kripke-indexing to track the collection of datums currently being
  manipulated by the program.
\item Monotonicity coeffects that track whether a program uses inputs
  co-, contra-, or in-variantly (or not at all).
\item Sensitivity typing, tracking the sensitivity of programs in
  terms of input changes. This forms the core of systems for
  differential privacy \cite{reed10distance}.
\item Information flow typing, in the style of the Dependency Core
  Calculus \cite{abadi99core}.
\end{enumerate}
Through discussion at the workshop, we hope to discover more
applications of our framework. In future work, we plan to extend our
framework with type dependency, and to explore the space of inductive
data types and elimination principles possible in the presence of
usage information.

The syntax and semantics we present here have been formalised in Agda:
\url{https://github.com/laMudri/quantitative/}. Formalisation of the
examples is in progress.