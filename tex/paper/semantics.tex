Resources will provide a refinement of the semantics of typed programs.
More precisely, we provide a standard semantics of merely well \emph{typed}
terms into $\mathrm{Set}$, which allows us to interpret terms as programs.
Then, our semantics of well \emph{resourced} terms is into indexed binary
relations over elements of the $\mathrm{Set}$ semantics.

In addition to the parametricity of the syntax over the choice of resource
annotations, the semantics is parametrised over a monoidal category of worlds,
which keep track of resources and restrictions we have accumulated.
Also, the choice of how to interpret the base type and the bang modality is
fairly liberal.

\subsection{Semantics of typed terms}

We begin with a standard semantics in $\mathrm{Set}$ of typed terms.
We do not require terms to be well resourced at this stage, and consequently
this semantics is entirely oblivious to the concerns of resourcing.
In particular, $\excl{\rho}{}$ has no effect, and the two types of products are
conflated.

\begin{displaymath}
  \begin{array}{c@{\hspace{0.5in}}c}
    \begin{eqns}
      \sem{\base} &=& X_\base \\
      \sem{\excl{\rho}{A}} &=& \sem{A} \\
      \sem{\tensorOne} &=& \{*\} \\
      \sem{\withTOne} &=& \{*\} \\
      \sem{\sumTZero} &=& \{\} \\
    \end{eqns}
    &
    \begin{eqns}
      \\
      \sem{\fun{A}{B}} &=& \sem{A} \rightarrow \sem{B} \\
      \sem{\tensor{A}{B}} &=& \sem{A} \times \sem{B} \\
      \sem{\withT{A}{B}} &=& \sem{A} \times \sem{B} \\
      \sem{\sumT{A}{B}} &=& \sem{A} \uplus \sem{B} \\
    \end{eqns}
  \end{array}
\end{displaymath}

Typing contexts $\Gamma$ are interpreted as cartesian products of their points.
Each typing derivation $\Gamma \vdash M : A$ is interpreted as a function
$\sem{M} : \sem{\Gamma} \to \sem{A}$ in the standard way, plus the following
clauses for the bang rules.

\begin{displaymath}
  \begin{eqns}
    \sem{\bang{M}}~\gamma &=& \sem{M}~\gamma \\
    \sem{\bm{C}{M}{x}{N}}~\gamma &=& \sem{N}~(\gamma, \sem{M}~\gamma)
  \end{eqns}
\end{displaymath}

Each constant $c$ must be given a semantics independent of context.

\subsection{Semantics of resourced terms}

The refinement resources provide on top of this is a Kripke-indexed logical
relation $\sem{A}^R : \mathcal{W}^{op} \to \operatorname{Rel}\sem{A}$ for each type $A$.
$\mathcal{W}$ is assumed to be a symmetric monoidal category.

\subsubsection{Worlds}

Inspired by modal logic \fixme{ref}, our relational semantics assumes a category
of worlds $\mathcal W$.
The world keeps track of what resources we have allocated to which parts of the
program.
To this end, we need some way of dividing a world into finitely many parts.
The stipulation that $\mathcal W$ is a \emph{symmetric monoidal}
category provides us with this.

This gives a semantic counterpart to tensor products.

\fixme{examples}

\subsubsection{Relation transformer}

\fixme{semantic bang}

\subsubsection{The semantics}

Let $\left| X \right|$ be the proposition that the set $X$ is inhabited.

\begin{displaymath}
  \begin{array}{rllcl}
    \sem{\base}^R & w & (c,c') &\iff& R_\base~w~(c,c') \\
    \sem{\fun{A}{B}}^R & w & (f,f')
                               &\iff& \forall x,y.~
                                 \left| \mathcal W(x, y \otimes w) \right|
                                 \Rightarrow \\
    &&&& \forall a,a'.~\sem{A}^R~y~(a,a')
                                 \Rightarrow \sem{B}^R~x~(f~a,f'~a') \\
    \sem{\excl{\rho}{A}}^R & w & (a,a') &\iff& \oc_\rho \sem{A}^R~w~(a,a') \\
    \sem{\tensorOne}^R & w & (*,*) &\iff& \left|\mathcal W(w, I)\right| \\
    \sem{\tensor{A}{B}}^R & w & ((a,b),(a',b'))
                               &\iff& \exists x,y.~
                                 \left| \mathcal W(w, x \otimes y) \right|
                                 \wedge \sem{A}^R~x~(a,a')
                                 \wedge \sem{B}^R~y~(b,b') \\
    \sem{\withTOne}^R & w & (*,*) &\iff& \top \\
    \sem{\withT{A}{B}}^R & w & ((a,b),(a',b')) &\iff&
    \sem{A}^R~w~(a,a') \wedge \sem{B}^R~w~(b,b') \\
    \sem{\sumT{A}{B}}^R & w & (\operatorname{inl} a,\operatorname{inl} a')
                               &\iff& \sem{A}^R~w~(a,a') \\
    \sem{\sumT{A}{B}}^R & w & (\operatorname{inr} b,\operatorname{inr} b')
                               &\iff& \sem{B}^R~w~(b,b') \\
  \end{array}
\end{displaymath}

Type-and-resourcing contexts $\ctx{\Gamma}{R}$ are interpreted as
tensor products of banged points.

%\begin{displaymath}
%  \begin{eqns}
%    \sem{\ctx{\cdot}{\cdot}}^R~w~(*,*) &=& \mathcal W(w, I) \\
%    \sem{\ctx{\Gamma}{R}, \ctxvar{x}{A}{\rho}}^R~w
%    ~((\gamma,a), (\gamma',a')) &=& 
%  \end{eqns}
%%  \sem{\ctxvar{x_1}{A_1}{\rho_1}, \ldots \ctxvar{x_n}{A_n}{\rho_n}} \coloneqq
%\end{displaymath}

\subsubsection{Fundamental lemma}

\begin{displaymath}
  \ctx{\Gamma}{R} \vdash t : T \implies \sem{\ctx{\Gamma}{R}}^R w~(\gamma, \gamma') \implies \sem{T}^R w~(\sem{t}\gamma, \sem{t}\gamma')
\end{displaymath}
