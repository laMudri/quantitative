\documentclass[fleqn]{beamer}

\usepackage{booktabs}
\usepackage{subcaption}

\usepackage{stmaryrd}
\usepackage{mathpartir}
\usepackage{amssymb}
\usepackage{cmll}
\usepackage{xcolor}
\usepackage{makecell}
\usepackage{tikz-cd}
\usepackage{nccmath}

\usetikzlibrary{positioning}
% \usetikzlibrary{decorations.markings}
\usetikzlibrary{arrows}

\def\newelims{1}
\def\multnotation{1}
\ifx\newelims\undefined
  \def\newelims{0}
\fi

\newcommand{\bind}[2]{\{#1\}#2}
%\newcommand{\ctx}[2]{{#1}^{\rescomment{#2}}}
\newcommand{\ctx}[2]{{#2}^{#1}}
\newcommand{\ctxvar}[3]{#1 \stackrel{\rescomment{#3}}: #2}
%\newcommand{\rescomment}[1]{{\color{red}#1}}
\newcommand{\rescomment}[1]{{#1}}


\newcommand{\ann}[2]{#1 : #2}
\newcommand{\emb}[1]{\underline{#1}}

\newcommand{\base}[0]{\iota}

\newcommand{\fun}[2]{#1 \multimap #2}
\newcommand{\lam}[2]{\lambda #1.~#2}
\newcommand{\app}[2]{#1~#2}

\newcommand{\excl}[2]{\oc_{#1} #2}
\newcommand{\bang}[1]{\operatorname{bang} #1}
\newcommand{\bm}[4]{\if\newelims0%
\operatorname{bm}_{#1}(#2, \bind{#3}{#4})%
\else%
\mathrm{let}~\bang{#3} = #2~\mathrm{in}~#4 : #1%
\fi}

\newcommand{\tensorOne}[0]{1}
\newcommand{\unit}[0]{{*}_\otimes}
\newcommand{\del}[3]{\if\newelims0%
\operatorname{del}_{#1}(#2, #3)%
\else%
\mathrm{let}~\unit = #2~\mathrm{in}~#3 : #1%
\fi}

\newcommand{\tensor}[2]{#1 \otimes #2}
\newcommand{\ten}[2]{(#1, #2)_{\otimes}}
\newcommand{\prm}[5]{\if\newelims0%
\operatorname{pm}_{#1}(#2, \bind{#3, #4}{#5})%
\else%
\mathrm{let}~\ten{#3}{#4} = #2~\mathrm{in}~#5 : #1%
\fi}

\newcommand{\withTOne}[0]{\top}
\newcommand{\eat}[0]{{*}_{\with}}

\newcommand{\withT}[2]{#1 \with #2}
\newcommand{\wth}[2]{(#1, #2)_{\with}}
\newcommand{\proj}[2]{\operatorname{proj}_{#1} #2}

\newcommand{\sumTZero}[0]{0}
\newcommand{\exf}[2]{\operatorname{ex-falso}_{#1}(#2)}

\newcommand{\sumT}[2]{#1 \oplus #2}
\newcommand{\inj}[2]{\operatorname{inj}_{#1} #2}
\newcommand{\cse}[6]{\if\newelims0%
\operatorname{case}_{#1}(#2, \bind{#3}{#4}, \bind{#5}{#6})%
\else%
\mathrm{case}~#2~\mathrm{of}~\inj{0}{#3} \mapsto #4; \inj{1}{#5} \mapsto #6%
\fi}


\newcommand{\typed}[1]{\mathit{#1t}}
\newcommand{\resourced}[1]{\mathit{#1r}}


\newcommand{\sem}[1]{\llbracket #1 \rrbracket}

%\def\tobar{\mathrel{\mkern3mu  \vcenter{\hbox{$\scriptscriptstyle+$}}%
%                    \mkern-12mu{\to}}}
\newcommand\tobar{\mathrel{\ooalign{\hfil$\mapstochar\mkern5mu$\hfil\cr$\to$\cr}}}


% Beamer hacks

\makeatletter
\newcommand{\subst}[2][]{\ext@arrow 0359\Rightarrowfill@{#1}{#2}}

% Detect mode. mathpalette is used to detect the used math style
\newcommand<>\Alt[2]{%
  \begingroup
  \ifmmode
  \expandafter\mathpalette
  \expandafter\math@Alt
  \else
  \expandafter\make@Alt
  \fi
  {{#1}{#2}{#3}}%
  \endgroup
}

% Un-brace the second argument (required because \mathpalette reads the three arguments as one
\newcommand\math@Alt[2]{\math@@Alt{#1}#2}

% Set the two arguments in boxes. The math style is given by #1. \m@th sets \mathsurround to 0.
\newcommand\math@@Alt[3]{%
  \setbox\z@ \hbox{$\m@th #1{#2}$}%
  \setbox\@ne\hbox{$\m@th #1{#3}$}%
  \@Alt
}

% Un-brace the argument
\newcommand\make@Alt[1]{\make@@Alt#1}

% Set the two arguments into normal boxes
\newcommand\make@@Alt[2]{%
  \sbox\z@ {#1}%
  \sbox\@ne{#2}%
  \@Alt
}

% Place one of the two boxes using \rlap and place a \phantom box with the maximum of the two boxes
\newcommand\@Alt[1]{%
  \alt#1%
  {\rlap{\usebox0}}%
  {\rlap{\usebox1}}%
  \setbox\tw@\null
  \ht\tw@\ifnum\ht\z@>\ht\@ne\ht\z@\else\ht\@ne\fi
  \dp\tw@\ifnum\dp\z@>\dp\@ne\dp\z@\else\dp\@ne\fi
  \wd\tw@\ifnum\wd\z@>\wd\@ne\wd\z@\else\wd\@ne\fi
  \box\tw@
}

\makeatother

\title{Linear metatheory via linear algebra}
\author{Bob Atkey\inst{1} \and James Wood\inst{1}}
\institute{\inst{1}University of Strathclyde}
\date{TYPES, June 2019}

\begin{document}
\setlength{\abovedisplayskip}{0pt}
\frame{\titlepage}
\begin{frame}
  \frametitle{Introduction}
  \begin{itemize}
  \item<1-> $\ctx{\Gamma}{\uncover<3->R} \vdash M : A$
  \item<2-> $\ctxvar{x_1}{A_1}{\uncover<3->{\rho_1}}, \ldots,
    \ctxvar{x_n}{A_n}{\uncover<3->{\rho_n}} \vdash M : A$
  \item<4-> The variable rule:
    \[
      \inferrule*[right=Var]
      {\uncover<5->{\rescomment 0, \ldots, \rescomment 0, \rescomment 1,
          \rescomment 0, \ldots, \rescomment 0 \subres \rescomment{\rho_1},
          \ldots, \rescomment{\_}, \rescomment{\pi},
          \rescomment{\_}, \ldots, \rescomment{\rho_n}}}
      {\ctxvar{x_1}{A_1}{\uncover<5->{\rho_1}}, \ldots,
        \ctxvar{y}{B}{\uncover<5->{\pi}}, \ldots,
        \ctxvar{x_n}{A_n}{\uncover<5->{\rho_n}} \vdash y : B}
    \]
  \item<6-> Examples:
    \setbeamercovered{transparent}
    \begin{enumerate}
    \item<7-> $\ctxvar{x}{A}{1}, \ctxvar{y}{B}{0} \vdash x : A$ (linearity)
    %\item<8-> $\ctxvar{x}{\mathbb Z}{\uparrow}, \ctxvar{y}{\mathbb Z}{\downarrow}
    %  \vdash \bang y : \excl{\downarrow}{\mathbb Z}$ (monotonicity)
    \item<8-> $\ctxvar{x}{\mathbb Z}{\uparrow}, \ctxvar{y}{\mathbb Z}{\downarrow}
      \vdash x - \bang y : \mathbb Z$ (monotonicity)
    \item<9-> $\ctxvar{x}{\mathrm{Secret}}{\mathrm{prv}}
      \nvdash \_ : \excl{\mathrm{pub}}{\mathrm{Secret}}$ (privacy)
    \end{enumerate}
  \end{itemize}
\end{frame}
\setbeamercovered{transparent}
\begin{frame}
  \frametitle{What's happening?}
  \begin{mathpar}
    \onslide<1,5>{
      % Sharing
      \inferrule*[right=$\&$-I]
      {\ctx{\Gamma}{R} \vdash M : A
        \\\\ \ctx{\Gamma}{R} \vdash N : B
      }
      {\ctx{\Gamma}{R} \vdash \wth{M}{N} : \withT{A}{B}}
    }
    \and
    \onslide<2,5>{
      % Splitting
      \inferrule*[right=$\otimes$-I]
      {\ctx{\Gamma}{P} \vdash M : A
        \\\\ \ctx{\Gamma}{Q} \vdash N : B
        \\\\ \subrctx{\resctx P + \resctx Q}{\resctx R}
      }
      {\ctx{\Gamma}{R} \vdash \ten{M}{N} : \tensor{A}{B}}
    }
    \and
    \onslide<3,5>{
      % Scaling
      \inferrule*[right=$\excl{\rho}{}$-I]
      {\ctx{\Gamma}{P} \vdash M : A
        \\\\ \subrctx{\rescomment \rho \resctx P}{\resctx R}
      }
      {\ctx{\Gamma}{R} \vdash \bang{M} : \excl{\rho}{A}}
    }
    \and
    \onslide<4,5>{
      % Binding
      %\inferrule*[right=$\excl{\rho}{}$-E]
      %{\ctx{\Gamma}{P} \vdash M : \excl{\rho}{A}
      %  \\\\ \ctx{\Gamma}{Q}, \ctxvar{x}{A}{\rho} \vdash N : B
      %  \\\\ \subrctx{\resctx P + \resctx Q}{\resctx R}
      %}
      %{\ctx{\Gamma}{R} \vdash \bm{B}{M}{x}{N} : B}
      \inferrule*[right=$\to$-I]
      {\ctx{\Gamma}{R}, \ctxvar{x}{A}{\rho} \vdash M : B}
      {\ctx{\Gamma}{R} \vdash \lam{x}{M} : \rescomment\rho A \to B}
    }
  \end{mathpar}
\end{frame}
\setbeamercovered{invisible}
\begin{frame}
  \frametitle{Substitution}
    \[
      \inferrule*[lab=Subst]
      {\ctx{\Gamma}{\uncover<2->{P}} \vdash M : A
        \\ {\begin{pmatrix}
            \rescomment{\uncover<2->{\Alt<-3>{\resctx Q}{\Alt<4>{\wn}{S_1}}}}\Delta
            \vdash N_1 : \Gamma_1 \\
            \vdots \\
            \rescomment{\uncover<2->{\Alt<-3>{\resctx Q}{\Alt<4>{\wn}{S_m}}}}\Delta
            \vdash N_m : \Gamma_m
          \end{pmatrix}}}
      {\ctx{\Delta}{\uncover<2->{Q}} \vdash M\{N\} : A}
    \]
  \begin{itemize}
  \item<3->
    \begin{minipage}[t]{\linewidth}
      \begin{overprint}
        \onslide<3>
        Wrong! Try identity substitution on \\
        $\ctx{\Gamma}{P} = \ctx{\Delta}{Q} = \ctxvar{x}{B}{1}, \ctxvar{y}{C}{1}$.

        \onslide<4->
        We need to split $\resctx{Q}$ up.
      \end{overprint}
      \[
        \begin{pmatrix}
          \ctxvar{x}{B}{1}, \ctxvar{y}{C}{\alt<3>{1}{0}} \alt<3>\nvdash\vdash x
          : B \\
          \ctxvar{x}{B}{\alt<3>{1}{0}}, \ctxvar{y}{C}{1} \alt<3>\nvdash\vdash y
          : C
        \end{pmatrix}
      \]
    \end{minipage}
  \item<5->
    What is the relationship between $\resctx P$, $\resctx Q$, and
    $\rescomment{S}$?
  \end{itemize}
\end{frame}
\begin{frame}
  \frametitle{Substitution, stripped down}
    \[
      \inferrule*[lab=Subst]
      {\alt<-2>{\resctx P}{\rescomment{\pi_1}x_1, \ldots, \rescomment{\pi_m}x_m}\only<1>\Gamma \vdash M \only<1>{: A}
        \\ {\begin{pmatrix}
            \alt<-2>{\rescomment{S_1}}{\rescomment{\sigma_{1,1}}y_1, \ldots, \rescomment{\sigma_{1,n}}y_n}\only<1>\Delta \vdash N_1 \only<1>{: \Gamma_1} \\
            \vdots \\
            \alt<-2>{\rescomment{S_m}}{\rescomment{\sigma_{m,1}}y_1, \ldots, \rescomment{\sigma_{m,n}}y_n}\only<1>\Delta \vdash N_m \only<1>{: \Gamma_m} \\
          \end{pmatrix}}}
      {\alt<-2>{\resctx Q}{\rescomment{\rho_1}y_1, \ldots, \rescomment{\rho_n}y_n}\only<1>\Delta \vdash M\{N\} \only<1>{: A}}
    \]
  \begin{itemize}
  \item
    What is the relationship between $\resctx P$, $\resctx Q$, and
    $\rescomment{S}$?
  \item<4->
    $\rescomment{\rho_j}$ is a weighted sum of $\rescomment{\sigma_{-,j}}$
    according to $\rescomment{\pi_{-}}$.
  \item<5->
    \[
      \begin{pmatrix}
        \rescomment{\pi_1} & \cdots & \rescomment{\pi_m}
      \end{pmatrix}
      \begin{pmatrix}
        \rescomment{\sigma_{1,1}} & \cdots & \rescomment{\sigma_{1,n}} \\
        \vdots & \ddots & \vdots \\
        \rescomment{\sigma_{m,1}} & \cdots & \rescomment{\sigma_{m,n}} \\
      \end{pmatrix}
      \subres
      \begin{pmatrix}
        \rescomment{\rho_1} & \cdots & \rescomment{\rho_n}
      \end{pmatrix}
    \]
  \end{itemize}
\end{frame}
\begin{frame}
  \frametitle{What is a substitution?}
  \begin{itemize}
  \item Notation: $\rescomment 1x$ is the basis vector at $x$
  \item A simultaneous substitution
    $\ctx{\Delta}{Q} \subst{} \ctx{\Gamma}{P}$ is:
    \begin{itemize}
    \item A $\left| \resctx P \right|$-by-$\left| \resctx Q \right|$ matrix
      $\rescomment S$, such that
    \item For each $x : A$ in $\Gamma$, a term $N_x$ such that
      $(\rescomment 1x\rescomment S)\Delta \vdash N_x : A$
    \item $\resctx P \rescomment S \subres \resctx Q$
    \end{itemize}
  \end{itemize}
\end{frame}
\setbeamercovered{transparent}
\begin{frame}
  \frametitle{Substitution on \TirName{$\&$-I}}
  \[
    \inferrule*[right=Subst]{
      \inferrule*[right=$\&$-I]
      {\ctx{\Gamma}{P} \vdash M : A
        \\ \ctx{\Gamma}{P} \vdash N : B
      }
      {\ctx{\Gamma}{P} \vdash \wth{M}{N} : \withT{A}{B}}
      \\
      \ctx{\Delta}{Q} \subst{\sigma} \ctx{\Gamma}{P}
    }
    {\ctx{\Delta}{Q} \vdash \wth{M\{\sigma\}}{N\{\sigma\}} : \withT{A}{B}}
  \]
  \pause
  \[
    \inferrule*[right=$\&$-I]
    {
      \inferrule*[right=Subst]
      {
        \ctx{\Gamma}{P} \vdash M : A
        \\\\
        \ctx{\Delta}{Q} \subst{\sigma} \ctx{\Gamma}{P}
      }
      {\ctx{\Delta}{Q} \vdash M\{\sigma\} : A}
      \\
      \inferrule*[Right=Subst]
      {
        \ctx{\Gamma}{P} \vdash N : B
        \\\\
        \ctx{\Delta}{Q} \subst{\sigma} \ctx{\Gamma}{P}
      }
      {\ctx{\Delta}{Q} \vdash N\{\sigma\} : B}
    }
    {\ctx{\Delta}{Q} \vdash \wth{M\{\sigma\}}{N\{\sigma\}} : \withT{A}{B}}
  \]
\end{frame}
\begin{frame}
  \frametitle{Substitution on \TirName{$\otimes$-I}}
  \[
    \inferrule*[right=Subst]{
      \inferrule*[right=$\otimes$-I]
      {\ctx{\Gamma}{P} \vdash M : A
        \\ \ctx{\Gamma}{Q} \vdash N : B
        \\\\ \subrctx{\resctx P + \resctx Q}{\resctx R}
      }
      {\ctx{\Gamma}{R} \vdash \ten{M}{N} : \tensor{A}{B}}
      \\
      \ctx{\Delta}{R'} \subst{\sigma} \ctx{\Gamma}{R}
    }
    {\ctx{\Delta}{R'} \vdash \ten{M\{\sigma\}}{N\{\sigma\}} : \tensor{A}{B}}
  \]
  \pause
  \begin{displaymath}
    \begin{matrix}
      \resctx{P'} := \resctx P \rescomment S &
      \resctx{Q'} := \resctx Q \rescomment S \\
      \ctx{\Delta}{P'} \subst{\sigma} \ctx{\Gamma}{P} &
      \ctx{\Delta}{Q'} \subst{\sigma} \ctx{\Gamma}{Q}
    \end{matrix}
  \end{displaymath}
  \pause
  \begin{displaymath}
    \resctx{P'} + \resctx{Q'}
    \uncover<1-2,4->{= \resctx P \rescomment S + \resctx Q \rescomment S}
    \uncover<1-2,5->{= (\resctx P + \resctx Q) \rescomment S}
    \uncover<1-2,6->{\subres \resctx R \rescomment S}
    \uncover<1-2,3,7>{\subres \resctx{R'}}
  \end{displaymath}
  \[
    \inferrule*[right=$\otimes$-I]
    {\ctx{\Delta}{P'} \vdash M\{\sigma\} : A
      \\ \ctx{\Delta}{Q'} \vdash N\{\sigma\} : B
      \\ \subrctx{\resctx{P'} + \resctx{Q'}}{\resctx{R'}}
    }
    {\ctx{\Delta}{R'} \vdash \ten{M\{\sigma\}}{N\{\sigma\}} : \tensor{A}{B}}
  \]
\end{frame}
\begin{frame}
  \frametitle{Substitution on \TirName{$\oc_{\rescomment\rho}$-I}}
  \[
    \inferrule*[right=Subst]{
      \inferrule*[right=$\oc_{\rescomment\rho}$-I]
      {\ctx{\Gamma}{P} \vdash M : A
        \\ \subrctx{\rescomment \rho \resctx P}{\resctx R}
      }
      {\ctx{\Gamma}{R} \vdash \bang{M} : \excl{\rho}{A}}
      \\
      \ctx{\Delta}{R'} \subst{\sigma} \ctx{\Gamma}{R}
    }
    {\ctx{\Delta}{R'} \vdash \bang{M\{\sigma\}} : \excl{\rho}{A}}
  \]
  \pause
  \begin{displaymath}
    \begin{matrix}
      \resctx{P'} := \resctx P \rescomment S \\
      \ctx{\Delta}{P'} \subst{\sigma} \ctx{\Gamma}{P}
    \end{matrix}
  \end{displaymath}
  \pause
  \begin{displaymath}
    \rescomment \rho \resctx{P'}
    = \rescomment \rho (\resctx P \rescomment S)
    = (\rescomment \rho \resctx P) \rescomment S
    \subres \resctx R \rescomment S
    \subres \resctx{R'}
  \end{displaymath}
  \[
    \inferrule*[right=$\oc_{\rescomment\rho}$-I]
    {\ctx{\Delta}{P'} \vdash M\{\sigma\} : A
      \\ \subrctx{\rescomment \rho \resctx{P'}}{\resctx{R'}}
    }
    {\ctx{\Delta}{R'} \vdash \bang{M\{\sigma\}} : \excl{\rho}{A}}
  \]
\end{frame}
\begin{frame}
  \frametitle{Substitution on \TirName{$\to$-I}}
  \[
    \inferrule*[right=Subst]{
      \inferrule*[right=$\to$-I]
      {\ctx{\Gamma}{R}, \ctxvar{x}{A}{\rho} \vdash M : B}
      {\ctx{\Gamma}{R} \vdash \lam{x}{M} : \rescomment\rho A \to B}
      \\
      \ctx{\Delta}{R'} \subst{\sigma} \ctx{\Gamma}{R}
    }
    {\ctx{\Delta}{R'} \vdash \lam{x}{M\{\sigma,x \mapsto x\}}
      : \rescomment\rho A \to B}
  \]
  \pause
  \begin{mathpar}
    \rescomment{S'} := \begin{pmatrix}
      & & & \rescomment 0 \\
      & \rescomment S & & \vdots \\
      & & & \rescomment 0 \\
      \rescomment 0 & \cdots & \rescomment 0 & \rescomment 1
    \end{pmatrix}
    \and
    \begin{pmatrix}\resctx R & \rescomment\rho\end{pmatrix}\rescomment{S'}
    = \begin{pmatrix}\resctx{R'} & \rescomment\rho\end{pmatrix}
  \end{mathpar}
  \pause
  \[
    \forall(z : C) \in \Gamma, x : A.~\exists N_z.~
    (\rescomment 1z\rescomment{S'})\Delta \vdash N_z : C
  \]
  \pause
  Cases:
  \begin{itemize}
  \item
    $
    \forall(z : C) \in \Gamma.~\exists N_z.~
    (\rescomment 1z\rescomment{S}) \Delta, \ctxvar{x}{A}{0} \vdash N_z : C
    $
  \item
    $
    \begin{pmatrix}\rescomment 0 & \cdots & \rescomment 0\end{pmatrix} \Delta,
    \ctxvar{x}{A}{1} \vdash x : A
    $
  \end{itemize}
\end{frame}
\begin{frame}[plain]
  \frametitle{Thank you}
  Questions?
\end{frame}
\begin{frame}
  \frametitle{Substitution on \TirName{Var}}
  \[
    \inferrule*[right=Subst]{
      \inferrule*[right=Var]
      {(x : A) \in \Gamma \\ \rescomment 1x \subres \resctx R}
      {\ctx{\Gamma}{R} \vdash x : A}
      \\
      \ctx{\Delta}{R'} \subst{\sigma} \ctx{\Gamma}{R}
    }
    {\ctx{\Delta}{R'} \vdash \sigma_x : A}
  \]
  \begin{displaymath}
    \rescomment 1x \rescomment S
    \subres \resctx R \rescomment S
    \subres \resctx{R'}
  \end{displaymath}
  From $\sigma$, we have
  $(\rescomment 1x\rescomment S)\Delta \vdash \sigma_x : A$. 
\end{frame}
\begin{frame}
  \frametitle{Single substitution}
  \[
    \inferrule*[right=SingleSubst]{
      \ctx{\Gamma}{P}, \ctxvar{x}{A}{\rho} \vdash M : B
      \\ \ctx{\Gamma}{Q} \vdash N : A
    }
    {(\resctx P + \rescomment\rho\resctx Q)\Gamma \vdash M[N/x]}
  \]
  \[
    \resctx P + \rescomment\rho\resctx Q \subst{[N/x]} \resctx P, \rescomment\rho x
  \]
  Need $\begin{pmatrix}\resctx P & \rescomment\rho\end{pmatrix}\rescomment S
  \subres \resctx P + \rescomment\rho\resctx Q$
  \[
    S := \begin{pmatrix}
      \rescomment 1 & \rescomment 0 & \cdots & \rescomment 0 \\
      \rescomment 0 & \rescomment 1 & \cdots & \rescomment 0 \\
      \vdots & \vdots & \ddots & \vdots \\
      \rescomment 0 & \rescomment 0 & \cdots & \rescomment 1 \\
      \resctx Q_1 & \resctx Q_2 & \cdots & \resctx Q_n
    \end{pmatrix}
  \]
\end{frame}

\end{document}
